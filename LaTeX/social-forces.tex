\section{Social force models}
\label{sec:social-forces}
One would think that the motion and dynamics of a human crowd would be 
governed by complex human decision making. However, the idea of social force 
models is modelling the behaviour using only a set of simple forces to 
describe the behaviour of the human pedestrians that comprise a crowd.

This works by calculating a set of social forces affecting each pedestrian, or 
agent, in the system. These forces are not real physical forces in the sense 
that they follow Newton's laws of motion, but rather a measure of the agents' 
motivation for acting in specific ways. However the name \emph{social forces} 
is not accidental: both notation and to some extent the interpretation is 
borrowed from physics. 

Since social forces have so much in common with physical forces, it makes 
sense to go into a bit more detail about what separates the two.
%TODO: Expand on the difference between social and physical forces.

\subsection{Variants of social force models}
% TODO: This should contain an overview of the different social force models, 
% starting with the one we picked to simulate, and explaining briefly how the 
% others differ. With references for all of them.

\subsection{Results obtained from the models}
% TODO: Expand the descriptions so they make sense to a reader that doesn't 
% know them.
% Add an introduction detailing the two main scenarios and where they come 
% from and why they are there.

\subsubsection{Lane Formation}
Especially in counterflow situation, pedestrians form lanes of walking 
directions, which is thought as a result of the non-linear body interaction 
force.

\subsubsection{Oscillatory Flows at Bottlenecks}
In bidirectional flow situations, each direction takes turns to pass the 
bottleneck.

\subsubsection{Freezing by Heating Effect}
This phenomenon is characterized by a blockage that follows an increasing 
density. The precondition for this effect is the driving force from desired 
velocity and actual velocity, while the tangential term from the interaction 
force is not needed.

\subsubsection{Faster-Is-Slower Effect}
It has been observed in evacuations situations that some process takes more 
time if it is performed at high speed. In our simulation, in order to see this 
effect, we need to have the body repulsion and friction force.
