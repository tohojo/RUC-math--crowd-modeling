% vim:ft=tex
\section{Analysis of the model}
\subsection{Social force models}
This is the section that malik is writing about Social force models in general
\clearpage

\subsection{Explanation of the model}
In this section we will go through the model from the article \cite{self-org} in great detail. 
First of all,  as stated briefly earlier this model is an agent based social force model. 
This means that the model uses individual entities to say something about the crowd as a whole. 
Each entity or agent is acted on by a collection of different forces. These forces can be
either repulsive or attractive. The general approach of the model is fairly simple. An agent
$\alpha$ wants to go in a desired direction with a desired speed. However the environment 
and other agents might force agent $\alpha$ to stray from the desired direction or have speeds 
different from the desired one. The change in position of agent $\alpha$ is given by:

	\begin{equation}
		\frac{d \vec{r_{\alpha}}}{dt} = \vec{V_{\alpha}} \left( t \right)
	\end{equation}

And, as we know from newtonian physics, the acceleration of agent $\alpha$ is 
then given by the summation of all the forces acting on the agent:

\begin{equation}
    \frac{d \vec{V_{\alpha}}}{dt} = \vec{f_{\alpha}} \left( t \right) + 
    \vec{\xi_{\alpha}}\left( t \right)
\end{equation}

here $\vec{\xi_{\alpha}} \left( t \right)$ is a random fluctuation of agent $\alpha$. This
force is there to incorporate the fact that fact the identical initial condition
generally will not lead to the same series of events. $\vec{f_{\alpha}} \left( t \right)$ 
is a summation of all the forces action in agent $\alpha$ from the environment 
and other agents. More specifically it is given by:

\begin{equation}\label{model}
    \vec{f_{\alpha}} = \vec{f^{0}_{\alpha}}\left( \vec{V_{\alpha}} \right) + 
    \vec{f_{\alpha B}} \left( \vec{r_{\alpha}} \right) +
    \sum_{\beta \neq \alpha} \vec{f_{\alpha \beta}} \left(\vec{r_{\alpha}}, 
    \vec{V_{\alpha}}, \vec{r_{\beta}}, \vec{V_{\beta}} \right) +
    \sum_{i} \vec{f_{\alpha i}} \left( \vec{r_{\alpha}}, \vec{r_{i}}, t 
    \right)
\end{equation}

So this is a summation of four different kinds of forces. We will go through 
them one at a time explaining their role in the model and their mathematical 
structure. The first term on the right hand side is a velocity dependent force 
and it is given by:

\begin{equation}
	\vec{f^{0}_{\alpha}}\left( \vec{V_{\alpha}} \right) =
    \frac{1}{\tau}
    \left( V_{\alpha}^{0} \vec{e_{\alpha}} - \vec{V_{\alpha}} \right)
\end{equation}

Here $\tau$ is the relaxation time. $\vec{V_{\alpha}}$ is the 
current velocity of the agent and $V_{\alpha}^{0}$ is the initial speed, that is 
the speed of the agent at the end of the last simulation step. $V_{\alpha}^{0}$ is given by:

\begin{equation}
    V_{\alpha}^{0} = \left[ 1 - \eta_{\alpha} \left( t \right) \right] 
    V_{\alpha}^{0} \left( 0 \right) +
    \eta_{\alpha} \left( t \right)V_{\alpha}^{\text{max}}
\end{equation}

Here $V_{\alpha}^{0} \left( 0 \right)$ is the velocity at the beginning of the 
first simulation step and $V_{\alpha}^{\text{max}}$ is the desired speed of agent
$\alpha$, that is the speed that agent $\alpha$ will try to get if it is allowed by the 
environment and other agents. $V_{\alpha}^{\text{max}}$ can be exceeded. $\eta_{\alpha}$ 
is called the impatience or nervousness of the agent and is given by:

\begin{equation}
	\eta_{\alpha} \left( t \right) =
    1 - \frac{\overline{V}_{\alpha} \left( t \right)}
             {V_{\alpha}^{0} \left( t \right)}
\end{equation}

Here $\overline{V}$ is the average speed in the desired direction and as 
earlier $V_{\alpha}^{0} \left( 0 \right)$ is the speed at the beginning of the 
first calculation step of the simulation.

Now the second term on the right hand side of \eqref{model} is the forces acting on agent 
$\alpha$ from the walls of the room. The force is given by:

\begin{equation}
    \vec{f_{\alpha B}} \left( \vec{r_{\alpha}} \right) =
    - \nabla_{\vec{r_{\alpha}}} V_{B}
    \left( \| \vec{r_{\alpha}} - \vec{r_{B}^{\alpha}} \| \right)
\end{equation}

Here $\nabla_{\vec{r_{\alpha}}}$ is the gradient and $V_B$ is a repulsive 
potential. $ \| \vec{r_{\alpha}} - \vec{r_{B}^{\alpha}} \|$ is the distance 
from agent $\alpha$ to the nearest point of the nearest wall.

The repulsive potential describes the force added from the walls to pedestrian $\alpha$, since $\alpha$ does not
want to get too close to the wall. So the closer $\alpha$ get to the wall the more the force from the wall gets.
The repulsion potential is given by  $V_{B} \left( \| \vec{r_{\alpha}} - \vec{r_{B}^{\alpha}} \| \right) =
V^0_{\alpha B} e^{- \| \vec{r_{\alpha}} - \vec{r_{B}^{\alpha}} \| / R }$

Here $V^0_{\alpha B}$ is a constant and $R$ is the radius of a pedestrian.

The reason for this repusion force from the wall is that the pedestrians do not want to get hurt by running into the walls
or get crushed between the panicing crowd and the wall [Helbing and Molnár, 1995].

The third term on the right hand side of \eqref{model} is a summation of all the 
force between agent $\alpha$ and agent $\beta$. It is a function of the position vector and the velocity of 
both agents, and it is given by:

\begin{equation}
    \sum_{\beta \left( \neq \alpha \right)}
        \vec{f_{\alpha \beta }}\left( t \right) =
        A_{\alpha}^{1} exp \left(
            \frac{ r_{\alpha \beta} - d_{\alpha \beta }}
                 {B_{\alpha}^1}
        \right)
    \vec{\eta_{\alpha \beta}} \cdot
    \left(
        \lambda_{\alpha} + \left(
            1 - \lambda_{\alpha}
        \right)
    \right) +
    A_{\alpha}^{2} exp\left(
        \frac{r_{\alpha \beta} - d_{\alpha \beta}}
             {B_{\alpha}^{2}}
    \right)
    \vec{\eta_{\alpha \beta}}
    \label{agentinteraction}
\end{equation}

Here $A_{\alpha}^{1}$, $A_{\alpha}^{2}$, $B_{\alpha}^{1}$, $B_{\alpha}^{2}$ 
and $\lambda_{\alpha}$ are all constants that can differ for each agent. 
$r_{\alpha \beta}$ is the sum of the radii of $\alpha$ and $\beta$ that is 
$r_{\alpha \beta} = r_{\alpha} + r_{\beta}$. $d_{\alpha \beta}$ is the 
distance from the center of mass of agent $\alpha$ and the center of mass of 
agent $\beta$ and is therefore given by $d_{\alpha \beta} = 
\|\vec{X_{\alpha}}\left( t \right) - \vec{X_{\beta}}\left( t \right) \|$. Here 
$\vec{X_{\alpha}}\left( t \right)$ amd $\vec{X_{\beta}}\left( t \right)$ are 
of course the vectors pointing to the center of mass of respectively agent 
$\alpha$ and $\beta$ at the time t. $\eta_{\alpha \beta}$ is the normal vector 
pointing from $\alpha$ to $\beta$ and it is given by:

\begin{equation}
    \eta_{\alpha \beta} =
        \frac{\vec{X_{\alpha}}(t) - \vec{X_{\beta}}(t)}
             {\|\vec{X_{\alpha}}(t) - \vec{X_{\beta}}(t) \|}
\end{equation}

the angle $\phi$ in \eqref{agentinteraction} is the angle between the normal 
vector pointing from agent $\beta$ to $\alpha$ and the direction in which 
agent $\alpha$ is moving. Cosine to the angle is $\cos \left( \phi 
\right)\left( t \right) = - \vec{\eta_{\alpha \beta}}\left( t \right) \cdot 
\vec{e_{\alpha}}\left( t \right)$.

Equation \eqref{agentinteraction} is divided into two terms. The first term on 
the right hand side reflects the agents tendency to stay at a certain distance 
from other agents. This part of the force is called the private sphere because 
the agent prefers to have some free space around him if possible. The radius 
of the private sphere can differ from agent to agent. The constant 
$A_{\alpha}^{1}$, $B_{\alpha}^{1}$ and $\lambda_{\alpha}$ control the nature 
of the private sphere $A_{\alpha}^1$ and $B_{\alpha}^1$ control the strength 
and range of the interaction respectively. $\lambda_{\alpha}$ is there to take 
into account a persons tendency to focus on things happening in front of him 
rather than behind him.

The fourth and last term in \eqref{model} represents the force from attraction 
in the room. Attractions can be either be either interesting sculptures or 
sights or familiar persons the agent prefer to be close to, such as friends 
and family. The mathematical structure of this force is the same as the force 
from other agents, however it is opposite in algebraic sign and has different 
constants. 

\begin{center}
\begin{tabular}{lll}
\hline
\multicolumn{3}{|c|}{\emph{List of constants and variables}}\\
\hline
\small{\textbf{Symbol}} & \small{\textbf{Description}} & \small{\textbf{Unit}}\\
\hline
$A_{\alpha}^{1}$ & \small{Controls the strength of the personal space force}\\
\hline
$A_{\alpha}^{2}$ & \small{Controls the strength of physical collisions}  & \\
\hline
$B_{\alpha}^{1}$ & \small{Controls the range of the personal space force} & \\
\hline
$B_{\alpha}^{2}$ & \small{Controls the range of physical collisions} & \\
\hline
$\lambda_{\alpha}$ & The anisotropic character of pedestrian interaction & \\
\hline
$\vec{f_{\alpha}} \left( t \right)$ & All forces acting on agent $\alpha$  & \\
\hline
$\vec{f_{\alpha B}} \left( \vec{r_{\alpha}} \right)$ & Force on agent $\alpha$ from walls & \\
\hline
$\vec{f_{\alpha \beta}} \left( \vec{r_{\alpha}}, \vec{r_{\beta}}, \vec{V_{\alpha}}, \vec{V_{\beta}} \right)$ & Force on agent $\alpha$ from agent $\beta$ & \\
\hline
$\vec{f_{\alpha i}} \left( \vec{r_{\alpha}}, \vec{r_{i}}, t \right)$ & Force on agent $\alpha$ from attractions & \\
\hline
$V_{\alpha}^{0}$ & Initial speed of agent $\alpha$ & \\
\hline
$V_{\alpha}^{\text{max}}$ & Maximum speed of agent $\alpha$ & \\
\hline
$\vec{V_{\alpha}^{\text{0}}}$ & Desired velocity of agent $\alpha$ & \\
\hline
$\overline{V}_{\alpha}$ & Average speed in desired direction & \\
\hline
$\vec{e}_{\alpha}$ & Vector pointing in desired direction of agent $\alpha$ & \\
\hline
$\vec{r}_{\alpha}\left( t \right) $ & Vector pointing to position of agent $\alpha$ at time t & \\
\hline
$\vec{r}_{B}$ & Vector pointing to nearest point of wall & \\
\hline
$r_{\alpha}$ & Radius of agent $\alpha$ & \\
\hline
$r_{\beta}$ & Radius of agent $\beta$ & \\
\hline
$r_{\alpha \beta}$ & The sum of the radii of agent $\alpha$ and $\beta$ & \\
\hline
$\vec{X}_{\alpha}$ & Vector pointing to center of mass of agent $\alpha$ & \\
\hline
$\vec{X}_{\beta}$ & Vector pointing to center of mass of agent $\beta$ & \\
\hline
$d_{\alpha \beta}$ & Distance between center of mass of agent $\alpha$ and $\beta$ & \\
\hline
$\tau_{\alpha}$ & Relaxation time $\alpha$ and $\beta$ & \\
\hline
$\eta_{\alpha}$ & Impatience of agent $\alpha$ at time t & \\
\hline
$\vec{\eta}_{\alpha \beta}\left( t \right)$ & Normal vector pointing from $\alpha$ to $\beta$ at time t & \\
\hline
$\vec{\xi}\left( t \right)$ & Stochastic element & \\
\hline
$\phi_{\alpha \beta} \left( t \right)$ & Angle between agent $\alpha$ and $\beta$ & \\
\hline

\end{tabular}
\end{center}

\clearpage
\section{Discussion}\label{sec:discussion}
\subsection{Limitations of the model}

\subsection{The force at doorways}

\subsection{Discussion on walls in special cases.}
In the general case of the repulsive force on a pedestrian, $\alpha$, from a wall nearby is given as a function of the vector from the nearest point. This point we calculate by finding the point that makes the vector form $\alpha$ to the wall be perpendicular to to vector that is the wall. In some cases though the point won't be on the wall it self. This of course makes no sense since you would then be repulsed by a non existing part of the wall meaning that you would avoid free areas which makes no sense. In this case you would have to use the end point of the wall. But doing this can make some unrealistic behavior as well, if the walls have the right composition. 

Let's start out by looking at a case with no problem. A case with no problems is a room where the angles between the walls is less than $180^o$, i.e. a squared room where they are $90^o$. For a pedestrian close to the corner between two walls, you would calculate the repulsive force from both of the walls. This you do in order for the pedestrian to avoid going through either one of the walls. When you do this you get a force directly away from each of the walls. This clearly makes sense and there is no problem in doing so.

\begin{figure}
\centering
\includegraphics[scale=0.5]{figures/Thewall.pdf}
\caption{Her you can se two walls joint together with an angle greater $180^0$ when seen from below. In the area `A` a pedestrian will only be perpendicular the first wall. In `B` there is no points perpendicular to any of the walls. And in `C` it is only the second wall that har points perpendicular to the pedestrians.  }
\label{fig:wallcase}
\end{figure}

This case were the angle between two walls is greater than $180^o$ could on the other hand give some problems if not handeled correctly. The case is sketched in figure \ref{fig:wallcase}. Here the are 3 different areas that a pedestrian $\alpha$ can be in. The area A where $\alpha$ is only perpendicular to wall $1$, in area B, $\alpha$ will not be perpendicular to any of the walls and in C he will be only perpendicular to wall 2. If a pedestrian is in area B then we would calculate the forces from the end point of the walls. This will be from the point where the two walls meet together. This will give you a double repulsion from one point and that doesn't make sense. Also when you are in are A or C you would get a repulsive force from a second wall you would be of no risk of going into and in many situations couldn't see because the first wall is blocking the sight. This of course doesn't make any sense too. So the way that we handle this situation is the following. When the angle between the walls is greater than $180^o$, from a pedestrian $\alpha$ point of view, you should look at the two walls as one, in the way that you will only calculate one force from the walls. In area A or C only the closest point on the closest wall should affect you. In the case of $\alpha$ being in area B the walls themselves   doesn't matter, only the vector going from the conjoint point of the walls to $\alpha$, should affect and only one time. Doing this, there should be no unrealistic scenarios.


\subsection{The repulsive force between agents in $ \Re ^{3}$}
From the given formula for calculating the repulsive force between agents in the description of the model, the part calculating the force to keep the personal space can be omitted when the agents are rather close to each other, then the calculation can be reduced as Equation (\ref{eq:re}).

\begin{equation}\label{eq:re}
\overrightarrow{f_{\alpha\beta}}(t) = A_{\alpha}^{2} exp\left[ \frac{r_{\alpha\beta} - d_{\alpha}\beta}{B_{\alpha}^{2}}\right]  \overrightarrow{n_{\alpha\beta}}
\end{equation}

Taking the norms of both sides of Equation (\ref{eq:re}), we can draw the relation between the value of 
$\overrightarrow{f_{\alpha\beta}}(t)$ and $d_{\alpha \beta}$, as in Figure (\ref{fig:physicalinteraction})
\\
\begin{figure}
\centering
\includegraphics[scale=0.45]{Figures/physicalinteraction.pdf} 
\caption{The function about the interaction force $\vec{f_{\alpha\beta}}(t)$ and the distance between two agents
$d_{\alpha\beta}$ }\label{fig:physicalinteraction}
\end{figure}

There is one intersection of the graph and the axis $ \left( 0, A_{\alpha}^{2} exp\left( \frac{r_{\alpha\beta} }{B_{\alpha}^{2}}\right)  \right)  $. If put into the constants, we will be able to get a maximum value of $ f_{\alpha\beta}(t) $, since the distance between agents cannot be negative. Here we set $ A_{\alpha}^{2} = 3 m/s^{2} $, $ r_{\alpha\beta} = 0.6 m $, and $ B_{\alpha}^{2} = 0.2 m $, so $ f_{\alpha\beta}(t)^{max} \doteq 60 m/s^{2} $, which is about six times the gravitational acceleration and represents a rather large force between agents (as large as six person's weight). \\\\
However, we notice that the effective part of the force calculated above is only the horizontal 
component that enables the agent to move horizontally in the plane where we do the simulation, 
but the reality is that the agents sometimes are also able to move vertically, for example, 
by stepping upon other people when they cannot take the pushing force from the surrounding agents. 
When that happens, the horizontal component of the repulsive force becomes smaller even if 
$d_{\alpha\beta}$ is kept the same.	
Therefore, a qualitative modification of dependence between $ f_{\alpha\beta}(t) $ and $ d_{\alpha\beta} $ could be:
\begin{figure}[hb]   
\centering
    {\includegraphics[scale=0.35]{Figures/ForceOverlapping.pdf}} 
    \caption{}
    \label{forceoverlapping}
\end{figure}
\\
\subsection{Use social force in further calculation}
use the value of forces to predict, as they are partly not real forces, the measurement does not reflect the reality in some range.
Pressure

