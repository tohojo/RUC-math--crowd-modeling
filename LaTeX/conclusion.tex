% vim:ft=tex
\section{Conclusion}
\label{sec:conclusion}

To answer the problem formulation, we have worked with the 
concept of social force models in thorough details. 

Based on our analysis of the model, we have constructed a computer simulation of
the model analogou to what is presented in section~\ref{sec:social-forces}.
Our implementation of the simulation is discussed in section~\ref{sec:simulation}, 
including set up of parameters and initial conditions, 
the structure of our simulation and how we obtain results from it.

We have been able to get a working simulation of the social force model 
presented in \cite{self-org} with components taken from \cite{ABconstant}. 
The most basic dynamics og the simulations seems to be running like as it 
should and our simulations show lane formation in bidirectional flow. We also  
see oscillatory flow at bottlenecks however freezin-by-heating 
and faster-is-slower effect is not observed at all.

We have discussed several possible cause for discrepancies in our model, and 
possible remedies. This can be narrowed down to two possibilities. 
One is that there is a lag of features in the model, like frictional forces and ect. 
The other possibility is the implementation of the model 
has been done wrongly and therefore errors occur throughout the simulations. 

\subsection{Further work}
Since the simulation we present in the report did't reproduce all the phenomena described in
section \ref{subsec:ThePhenomena}, if given more time our group would like to solve the discrepancies and add the possible solutions that has been talked about in section~\ref{sec:discussion}. The first thing we would implement into the model would be the frictional force, and make the frictional and repulsive forces velocity dependent. This should possibly give rise to the slow-is-faster effect and the freezing-by-heating effect, further more is should make the predictions made by the model more precise.
One could add path finding, enabling the model to simulate complex environments.

The interaction between the different parameters is interesting to learn more about. Our simulations shows us that changing one parameter, like the mean velocity, implies a change in the constants controlling interaction between pedestrians and interaction with walls. It is not at the moment completely clear how this interaction works. Also a more systematic work with the different constants could be interesting in learning the limitations of the model.  

It would be interesting to examen how large an error occur due to the euler approximation and if this error is of any significants to the predictions made by the model. If sow would some other method of approximation give rise to any significant improvements.

Another interesting investigation could consist of making predictions and evaluating actual designs such ass malls, theaters etc. Then investigate the predictions made by the model, compared to the fire marshals actual recommendations concerning the number of people allowed in the room, the number of fire exists, how much time should it take to clear the room, etc. Does the predictions made by the models correspond to real life recommendations?