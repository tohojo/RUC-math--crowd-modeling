\section{sec:discussion}\label{sec:discussion}
\subsection{Repulsive force}
In chapter 3 we have seen the formulas for calculating the repulsive forces from the wall and other agents are very similar, both are exponential functions decreasing with some distance, which makes us wonder if both come from the same origin.\\

\begin{itemize}
\item The difference between those two repulsive forces is basically one is from a single point and the other is from an object that has a dimension much larger than a single agent. If we imagine our agents are particles with negative charge, then there is a pair of repulsive forces which has direction along the line connecting the two particles and magnitude depending on the distance between them. Further we can put lots of particles along a bar, see Figure
Summing up all the repulsive forces from each particle on the bar will give the repulsive force that the charged bar on the single particle $\alpha$.

\item Applying the idea above to analyse our wall repulsion, as a simple start we place the agent $\alpha$ a distance $ d $ away from the wall which as a length $L$, and by symmetry we know the resulting force must point vertically downwards, see Figure.\\

Integrating the forces can be written as:

\begin{equation}\label{integ}
\vec{f_{\alpha B}} = 
\int_0^L \! \vec{f_{\alpha\beta}} \, \mathrm{d}l
\end{equation}

\item When the repulsive force is known, the expression for the wall potential can be calculated, which can be done by integrating the force along a path. Normally the potential at some position is defined as the work done on the particle by that potential field when the particle is moved from that position to infinitely far away.

\begin{equation}\label{pot}
	U_{B}= \int_{\vec{r_{\alpha}}�}^{text{f}\infty} \! \vec{f_{\alpha B}} \cdot \, \text{d}\vec{r_{\alpha}} 
\end{equation}

\item In some article [Helbing 2000], the repulsive force from the wall is given by:
\begin{equation}\label{helbing2000}
\vec{f_{\alpha B}} = 
	\left( 
			\left[ 
	A exp 
				\left[ 
						\left(  
							r_{\alpha}-d_{\alpha B}
						\right)  / B
				\right] +kg 
					\left( 
						r_{\alpha}-d_{\alpha B} 
					\right) 
			\right] 
		\right)
	\vec{n_{B}^{\alpha}}-\kappa g 
	\left(
		r_{\alpha}-d_{\alpha B}
	\right) 
	\left(
			\vec{V_{\alpha}}\cdot \vec{t_{iw}}
	\right) 
\vec{t_{iw}}
\end{equation}

where $ r_{\alpha} $,  $ \vec{V_{\alpha}} $ are the radius and velocity of agent $ \alpha $ as we noted before, $ A $, $ B $, $ k $, $ \kappa $ are constants, $ d_{\alpha B} $ is the distance from the agent to the wall, $ \vec{n_{B}^{\alpha}} $ is a normal vector from the wall, $ \vec{t_{iw}} $ is a tangent vector to the wall. $ g $ is a function of some variable.\\
In this expression, the wall repulsion force to agent $ \alpha $ is not always perpendicular to the wall, but the force has two components, one perpendicular to the wall and the other tangent to the wall. Particularly the tangent component depends on the velocity of $ \alpha $. However, in Helbing's latest articles he does not show the expression of Equation \ref{helbing2000}, but use the gradient of potential instead. 
\end{itemize}


\subsection{The repulsive force between agents in $ \Re ^{3}$}
From the given formula for calculating the repulsive force between agents in the description of the model, the part calculating the force to keep the personal space can be omitted when the agents are rather close to each other, then the calculation can be reduced as Equation (\ref{eq:re}).

\begin{equation}\label{eq:re}
\overrightarrow{f_{\alpha\beta}}(t) = A_{\alpha}^{2} exp\left[ \frac{r_{\alpha\beta} - d_{\alpha}\beta}{B_{\alpha}^{2}}\right]  \overrightarrow{n_{\alpha\beta}}
\end{equation}

Taking the norms of both sides of Equation (\ref{eq:re}), we can draw the relation between the value of 
$\overrightarrow{f_{\alpha\beta}}(t)$ and $d_{\alpha \beta}$, as in Figure (\ref{fig:physicalinteraction})
\\
\begin{figure}
\centering
\includegraphics[scale=0.45]{Figures/physicalinteraction.pdf} 
\caption{The function about the interaction force $\vec{f_{\alpha\beta}}(t)$ and the distance between two agents
$d_{\alpha\beta}$ }\label{fig:physicalinteraction}
\end{figure}

There is one intersection of the graph and the axis $ \left( 0, A_{\alpha}^{2} exp\left( \frac{r_{\alpha\beta} }{B_{\alpha}^{2}}\right)  \right)  $. If put into the constants, we will be able to get a maximum value of $ f_{\alpha\beta}(t) $, since the distance between agents cannot be negative. Here we set $ A_{\alpha}^{2} = 3 m/s^{2} $, $ r_{\alpha\beta} = 0.6 m $, and $ B_{\alpha}^{2} = 0.2 m $, so $ f_{\alpha\beta}(t)^{max} \doteq 60 m/s^{2} $, which is about six times the gravitational acceleration and represents a rather large force between agents (as large as six person's weight). \\\\
However, we notice that the effective part of the force calculated above is only the horizontal 
component that enables the agent to move horizontally in the plane where we do the simulation, 
but the reality is that the agents sometimes are also able to move vertically, for example, 
by stepping upon other people when they cannot take the pushing force from the surrounding agents. 
When that happens, the horizontal component of the repulsive force becomes smaller even if 
$d_{\alpha\beta}$ is kept the same.	
Therefore, a qualitative modification of dependence between $ f_{\alpha\beta}(t) $ and $ d_{\alpha\beta} $ could be:
\begin{figure}[hb]   
\centering
    {\includegraphics[scale=0.35]{Figures/ForceOverlapping.pdf}} 
    \caption{}
    \label{forceoverlapping}
\end{figure}
\\
\subsection{Use social force in further calculation}
use the value of forces to predict, as they are partly not real forces, the measurement does not reflect the reality in some range.
Pressure