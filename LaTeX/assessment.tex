\section{Assessment of social force models}
\label{sec:assessment}
In this section we will present our assessment of social force models in 
general, based on our work with the specific model and the results we have 
obtained. To expand the scope of the discussion, we will first outline various 
extensions to social force models that are presented elsewhere, which 
introduce some features into the models that are lacking in the version we 
have worked with. Based on our results and these extensions, we will then 
outline the advantages and weaknesses of social force models in general, and 
conclude by assessing the state of these types of models.

\subsection{Extensions of the model}
There are several features that might be added to the social force model we 
have worked with, that will widen the scope of scenarios for which the model 
is useful. As mentioned in section~\ref{sec:variants}, there already exists 
models that incorporate some of these features. In this section we will 
outline some of these features, to give a broader view of what is possible 
with social force-based models.

The features we will look at are \emph{path finding}, which will allow the 
model to simulate more complicated scenarios, and \emph{communication between 
pedestrians}, \emph{falling} and \emph{handling lowered visibility} that can 
add types of pedestrian behaviour that is not handled well in our version of 
the model.

\begin{itemize}
    \item \textbf{Path finding:} Our model does not incorporate path 
        finding, i.e. pedestrians only move towards one fixed target. Adding 
        path finding would allow the model to simulate people moving in more 
        complex environments than what is possible with our model. There 
        exists various models that incorporate this feature \cite{HiDAC}.
 
    \item \textbf{Communication between pedestrians:} Adding an ability for 
        pedestrians to communicate would enable us to make simulations of 
        cases where groups of pedestrians follow each other (such as a 
        family), and it would enable pedestrians to cooperate in finding 
        alternative routes. Models adding communication are also seen in the 
        literature \cite{HiDAC}.

    \item \textbf{Falling and injuries:} In situations of crowd panic, people 
        are often injured. This is either due to the enormous pressure that 
        can built up in a crowd or due to people falling and being trampled.  
        Our does not take into account these possibilities, but other models 
        do \cite{HiDAC}.

    \item \textbf{Lowered visibility:} Our model assumes that pedestrians can 
        identify their target and move towards it. This means that we cannot 
        simulate scenarios where pedestrians do not know in which direction 
        they want to move, e.g. a room filled with smoke. Adding alternative 
        mechanisms for pedestrian movement, such as having pedestrians close 
        to walls moving along them, would allow us to simulate pedestrian 
        movement in cases of lowered visibility \cite{HelbingNew}.
\end{itemize}

These features allows the social force models to simulate more complex cases 
than what is possible using our model, and since they have been incorporated 
into other models, we will include them in our assessment of the advantages 
and weaknesses  of social force models.

\subsection{Advantages of social force models}
The main advantage of using social force models for modelling crowds is 
expressed quite well in this quote from \cite{self-org}:

\begin{quote}
    The advantage of the social force-based simulation approach is its simple 
    form and its small number of parameters, which do not need to be 
    calibrated anew for each situation.
\end{quote}

That is, using this approach allows us to simulate a complex system using a 
quite simple formulation of the model. Simulating a large crowd of pedestrians 
using a classical (i.e. non-agent based) model would be very complex, and 
quite possibly impractical.

This simplification is possible, because the complexities of the model 
behaviour is moved from the formulation of the model and into the 
calculations. That is, a very large number of calculations are needed to get 
any meaningful results out of this model. This means that working with this 
kind of model would be impractical without the help of computers, and indeed a 
large quantity of processing power is necessary to get any results. In our 
simulations this has been most apparent in the need to implement the 
calculation-intensive parts of the model in the C programming language, to be 
able to achieve reasonable computation times.

Of course having a simple model that is practical to implement is of no use if 
it does not give useful results. Empirical studies have shown that social 
force models are able to show phenomena that correspond to real crowds, and as 
such do provide meaningful results in some cases. However, while parameters 
may not need to be calibrated for each situation, as the quote above says, 
calibration is required, and doing so is non-trivial. This means that we have 
not been able to identify parameters for our simulations that work across all 
the different scenarios.

\subsection{Weaknesses of social force models}
While social force models in some cases have been shown to give results that 
correspond to real life observations, there are several weaknesses to this 
approach to modelling crowds. Some of these weaknesses are related to the way 
the models are presented, and some are more fundamental to the nature of 
social force models.

When working with social force models, we have had to piece together a working 
model from several different sources. This exposes a difficulty in assessing 
the models: It is not always obvious if a given weakness is due to an inherent 
quality of the models, or if it is simply due to a weak or missing formulation 
of some part of it. Especially precise results of simulations have been 
difficult to find, and as has been shown in 
section~\ref{sec:model-to-simulation}, we have had to fill in several details 
ourselves. As mentioned above, the difficulty in estimating parameters has 
also provided a barrier in this respect.

Setting aside the difficulties in finding detailed information about the 
model, it is readily apparent that social force models are in a relatively 
early state of development. This means that the amount of empirical data 
available to assess the quality of the models' predictions is quite low. This 
means that even if the social force models have shown some promising results, 
it is impossible to say with confidence that they do indeed predict the actual 
behaviour of crowds very well.

Further compounding these uncertainties is the fact that the model is 
formulated in a way that is counter-intuitive to the way human behaviour is 
normally perceived. That human behaviour is reducible to simple repulsive 
forces is counter to what we believe is reasonable to expect. This means that 
if this is indeed the case, strong evidence is needed to convince us, which is 
not currently provided.

One of the reasons we are sceptic that these forces are able to completely 
explain human behaviour, is that they pertain to behaviour of objects in a 
physical world, but they do not obey the traditional physical laws of motion.  
It is clear that the idea of using forces to describe movement has its origins 
in physics, but excluding some of the properties of physical forces seems 
arbitrary.

Finally, the fact that different variants of social force models use 
completely different parameters and formulations of the different forces, 
sometimes even contradicting each other, makes it seem as though the models 
are changed in arbitrary ways to accommodate empirical observations. This 
belief is corroborated by the fact that the models do not make any new 
predictions that are then tested, but instead only seem to attempt to 
replicate already observed behaviour.

\subsection{Conclusion on the assessment}
Weighing the advantages and disadvantages of social force models against each 
other, it is quite apparent that the models do not instil a strong sense of 
confidence in their predictions. However, the crucial advantage that these 
models have, is that they are able to do \emph{some} predictions of crowd 
behaviour, that no other models have been able to. This means that while 
social force models are far from perfect, they are in many ways the best 
available way of evaluating e.g. a new building's suitability for efficient 
crowd movement. And while the models may not be able to provide any underlying 
mechanisms or reasons for crowd behaviour, in practice they may, given further 
adjustments and experiments with real life observations, be a substantial 
improvement over today's standards.
