% vim:ft=tex
\section{Simulation approach}
\label{sec:simulation}
In this section, we describe how our simulation is implemented, and how the 
implementation works.

Our simulation is implemented in the Python programming language, with the 
calculation intensive parts implemented in C for performance reasons. We 
assume a virtual coordinate system using meters as a base unit, and with the 
origin in the centre of the area we simulate. Each pedestrian (or actor) is 
described by a centre point and a radius, and each wall is described by a line 
segment connecting two points.

All parameters are stored as double precision floating point values where 
nothing else is indicated. We use custom data structures to keep track of the 
actors and walls while running the simulation. Python is used to set up the 
initial conditions, run the program's main control loop, and draw the 
simulation results through the \emph{PyGame} library \cite{pygame}. This 
allows to do both real-time animation as well as saving each simulation step 
to be assembled into a film afterwards. All calculations and data processing 
is done in a Python extension written in C, to increase performance.

\subsection{Initial conditions and constants}
When initialising the model, parameters are set for each pedestrian. In the 
model, every parameter can vary between actors, while in practice many of them 
do not. In this section, we go through the parameters and how they are set.  
For all random numbers, the operating system's built-in random number 
generator is used and considered to be sufficient for our purposes. We run 
multiple simulations of the same initial conditions by fixing the seed of the 
random number generator to the same value for each run. Distributions are 
drawn by using the distribution functions of the \emph{NumPy} mathematical 
library for Python \cite{numpy}.

\subsubsection{Position related parameters}
There are a number of parameters that are set that has to do with the initial 
position of actors and walls. They are:

\begin{itemize}
    \item \textbf{Wall endpoints:} Points describing the endpoints of the 
        walls. These are set according to the scenario we want to simulate, so 
        in a square room with a single exit in the middle of a wall, there 
        will be five wall segments (one on each side of the exit, and one for 
        each of the other walls).

    \item \textbf{Actor positions:} Each actor has a starting position 
        distributed randomly within the room. These are created by drawing a 
        set of random numbers for the x and y coordinates respectively, and 
        adjusting the range of this random number to be within the room's 
        dimensions. Actor positions are adjusted so that they do not overlap 
        with the walls by adjusting coordinates so that the distance from the 
        center to each wall is at most the radius. This adjustment is not made 
        between actors, so they may overlap initially. It is assumed that the 
        model will correct this within the first few simulation steps, which 
        is also what we have seen in practice.

    \item \textbf{Actor radii:} The actor radii are drawn from a normal 
        distribution with a mean of $0,2$ meters and a standard deviation of 
        $0,01$ meters. This is done to simulate a natural variety in physical 
        stature of humans, and to avoid deadlocks caused by perfectly 
        symmetrical forces that might otherwise occur \cite{helbing00}.
        %TODO: Check this reference, maybe better explanation?
\end{itemize}

\subsubsection{Movement related parameters}
A number of parameters are set to control the movement of the actors. They 
are:

\begin{itemize}
    \item \textbf{Target:} Each actor has a target that they move towards. 
        This target is set outside the exit actors will move towards, and is 
        the same for all actors when there is only one exit. In situations 
        where there are multiple exits, actors are set to move towards one of 
        the exits at random, regardless of their position within the room. 
        Since the model does not deal with pathfinding, targets are not 
        changed during the simulation. When an actor reaches its target, it is 
        considered to have escaped, and is removed from the simulation.

    \item \textbf{Initial velocity:} This is set as both a vector and a scalar 
        representing vector length. The scalar velocities are drawn from a 
        normal distribution with a mean of $1.34$ and a standard deviation of 
        $0,26$. The initial velocity vectors are created by multiplying the 
        scalar velocity with a normalised vector pointing from the actor's 
        initial position to the target.
        % TODO: Where do the mean and deviation come from?

    \item \textbf{Max speed:} TODO.

    \item \textbf{Desired velocity:} The desired velocity is the velocity the 
        actor wants to move at (see the explanation in 
        section~\ref{sec:the-model}). This is set equal to the initial velocity 
        under the assumption that when people start to leave a room they will 
        initially (try to) move at their desired velocity, and then be 
        affected by the model parameters once they start moving.

    \item \textbf{Relaxation time:} The relaxation time is the time it would 
        take an unhindered actor to return to their desired velocity after 
        having been hindered by something blocking their path. This is set to 
        one second for all actors.
        % TODO: Why one second, and is this explanation correct?

    \item \textbf{$\lambda$:} TODO.
\end{itemize}

\subsubsection{Constants}
The model includes a number of constants. These are parameters that do not 
vary between the agents, but are fixed for the whole simulation. They are:

\begin{itemize}
    \item \textbf{Timestep:} The timestep is the $\Delta T$ that passes for 
        each step of the simulation. As discussed in 
        section~\ref{sec:timestep}, there are various trade-offs in making 
        this parameter larger or smaller. We have experimented with different 
        values, and have found that a value of $0,01$ seconds make for a 
        simulation without errors such as jitter that results from larger 
        timestep values. Since setting the timestep corresponds to setting a 
        delta value for an Euler integration, there are various methods that 
        originate from this integration method, that might be used to vary the 
        timestep dynamically during the simulation. However, we have found 
        that with a fixed value of $0,01$ seconds, we get reasonable 
        performance of our simulation, so we have not found the need to 
        complicate our program by applying such methods.
        % TODO: Reference for dynamic timestep adjustment

    \item \textbf{$A_1$, $B_1$:} TODO.

    \item \textbf{$A_2$, $B_2$:} These values are given in \cite{helbing00}. 
        Although the model allows for them to vary between agents, we have 
        (just as is done in the article) set them to a fixed value for the 
        whole simulation. The values given are $A_2=3,0$ and $B_2 = 0,2$.
\end{itemize}

%The potential between pedestrian $\alpha$ and the wall $B$ is given by 

%\begin{equation}
%V_B=V_B^0 e^{-(r_\alpha - r_B^\alpha )/R} 
%\end{equation}
%where $V_B^0 = 10m^2s^{-2}$ and $R=0.2m$.
%These model parameters have been determined such that they are compatible with empirical data. 
%(Kilde: Dirk Helbing and Peter Molnar - Social force model for pedestrian dynamics). This is 
%the older article by Helbing, and it seems as if its the same initial conditions, as in 
%the article we are working with, except that the force between to pedestrians has changed. 
%But still I think we could use the old article to argue why these parameters 
%have the given values.\\

%\noindent
%$A^2_\alpha = 3m/s^2$ and $B^2_\alpha = 0.2 m\\$
%$A = 5 m/s^2$ and $B=0.1m$
%$r_{\alpha \beta} = 0.6m$
%$\lambda_a = 0.75$\\\\
%\noindent
%These values for the model have been calibrated with empirical data of pedestrian streams.
% TODO: What is this section doing here?

\subsection{The simulation steps}
The simulation is divided into two parts: Finding the accelerations (or 
resulting force) for all actors, and updating position and velocity for the 
actors.  Since the acceleration for each actor is dependent on both position 
and velocity of the other actors, splitting the calculations this way enables 
us to do the calculations of each actor in any order, and even parallel. The 
drawback is that the actors are only affected by the movement and positions of 
other actors as they were at the end of the last simulation step. This means 
that the time step has to be small enough that this doesn't matter in 
practice.

\subsubsection{Calculating the acceleration vectors}
The acceleration vectors for each actor, $\alpha$, is calculated as follows:

\begin{enumerate}
    \item Determine the acceleration vector provided by the actors' desired 
        direction of movement.
    \item For each other actor $\beta_1\dots\beta_n$, calculate the 
        acceleration vector provided by avoiding the actor $\beta_i$.
    \item Calculate acceleration vectors resulting from avoidance of the walls 
        $w_1\dots w_n$.
    \item Sum all the acceleration vectors to a resultant acceleration vector 
        $a$.
\end{enumerate}

Steps one to three correspond to the three parts of the model.

\subsection{Updating position and velocity}
After every actor has been updated with a resulting acceleration vector from 
the current simulation step, all actors update their position and velocity.  
The position is updated by calculating a displacement vector as follows:

\begin{equation}
    \Delta p = (v_x \Delta t + \frac{1}{2}a_x \Delta t^2, v_y \Delta t + 
    \frac{1}{2}a_y \Delta t^2)
\end{equation}

Where $v_x$ and $v_y$ are the $x$ and $y$ components of the velocity vector, 
$a_x$ and $a_y$ are the $x$ and $y$ components of the acceleration vector, and 
$\Delta t$ is the time step.

After updating the position, the actor's velocity is updated by adding the 
acceleration vector, to be used for the next simulation step.

% start parameters: p_n, v_n for each person.
% p for each wall
%
% steps:
%
% for each person:
%   calculate acceleration from the force parts of the model
%
% for each person:
%   update position by displacement vector
%       delta-p = (v_x*delta-t+1/2*a_x*delta-t^2, 
%       v_y*delta-t+1/2*a_y*delta-t^2)
%   update velocity for next step
%
%
% initial parameters:
%   Set using some sort of distribution around a mean
%
%
% constants:
%   Where do they come from?
