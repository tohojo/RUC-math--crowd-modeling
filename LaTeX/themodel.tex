\section{The model}
\label{sec:the-model}

\subsection{Explanation of our model}
In this section we will go through the model from the article \cite{self-org} in 
great detail, in the way that we understand the mathematical expressions and also 
see how those expressions represent the reality.\\

\underline{Force analysis:} \\
As the social force model is an agent based model, it focuses on looking at the 
quantities about an individual agent, at last get the motion of the crowd. In order 
to get the equation of motion we always start by analysing the forces acting on the object.\\\\

Here in this model, the agents cannot escape from Newton's three laws of motion, 
which mainly says that when there is a force it results a corresponding acceleration.  
Therefore, it is necessary to look at the forces.  However, our agent is not a ball 
being kicked around, it has a willingness to go to some destine place, and the 
"willingness" to go somewhere is hard to measure, but we know that the way an agent 
implement the will normally is by generating a static frictional force from the ground, 
and that force is here what we called a kind of "social force", which is 
$\vec{f^{0}_{\alpha}}$ in Figure \ref{ForceModel}. 

Some other major forces acting on the agent are the repulsive force from other agents 
namely $ \beta $ - the force called $ \vec{f_{\alpha\beta}} $ in Figure \ref{ForceModel}, 
and repulsive force from an obstacle (a wall for example) - the force called $ \vec{f_{\alpha B}} $ 
in Figure \ref{ForceModel}.  As these two repulsive forces are a kind of normal force, the direction 
should be perpendicular to the surface.

\begin{figure}[hb]
    \centering
    {\includegraphics[scale=0.45]{Figures/ForceModel.pdf}} 
    \caption[Notation of forces acting on an agent]{Illustration of the forces acting on an agent $ \alpha $. $ \beta $ is another agent and the grey bar on the top represents a wall. $ \vec{f_{\alpha\beta}} $ is the repulsive force from agent $ \beta $, and $ \vec{f_{\alpha B}} $ is the repulsive force from the wall. $ \vec{f^{0}_{\alpha}} $ is a force that represents agent $ \alpha $'s desire to reach the exit.
    The x and y axes are defined by the Cartesian coordinate system.}
    \label{ForceModel}
\end{figure}

So far in the model it only mentioned the forces on the horizontal level, the gravitational force and normal force from the ground which works vertically are not mentioned at all.  The reason is that in this model they only consider the motion on the horizontal plane.  Since there is no motion vertically, the gravitational force and the normal force from the ground cancel each other.\\\\
\underline{The equation of motion for agent $ \alpha $:}\\

The general approach of the model to get the equation of motion follows the standard way. Summing up the forces acting on $ \alpha $ gives the acceleration, which builds the equation of motion. It comes in steps as the following.\\
First of all, the equation of motion deals with the agent $ \alpha $'s position $ \vec{r_{\alpha}} $, velocity $ \vec{V_{\alpha}} $, and acceleration $ \vec{f_{\alpha}} $. As the agent is not just a mass point, it has a radius $ r_{\alpha} $. The position and velocity vectors and the radius are shown in Figure \ref{NotationOfAgent}.

\begin{figure}[hb]
    \centering
    {\includegraphics[scale=0.35]{Figures/NotationOfAgent.pdf}} 
    \caption[Notation of an agent]{Illustration of the visual presentation of the mathematical notations for position and velocity. As for agent $ \alpha $,
	    it has position vector $ \vec{r_{\alpha}} $, velocity vector $ \vec{V_{\alpha}} $, $\vec{e_{\alpha}}$ or $\vec{e_{\beta}}$ the normal vector pointing
	    to the exit and  $ r_{\alpha} $ or  $ r_{\beta} $ the radius of its body.
	    The x and y axes are defined by the Cartesian coordinate system.}
    \label{NotationOfAgent}
\end{figure}

The change in position $ \vec{r_{\alpha}} $ per time of 
agent $\alpha$ is actually the velocity $ \vec{V_{\alpha}} $:

\begin{equation}
		\frac{d \vec{r_{\alpha}}}{dt} = \vec{V_{\alpha}} \left( t \right)
\end{equation}

Also known from Newtonian physics, the change of velocity per time is the acceleration of agent $\alpha$, which is the result of a summation of all the forces acting on the agent, namely $\vec{f_{\alpha}} \left( t \right)$:

\begin{equation}
    \frac{d \vec{V_{\alpha}}}{dt} = \vec{f_{\alpha}} \left( t \right) 
\end{equation}

Referring to the part for force analysis, we need to add the driving force, repulsive force from the wall and repulsive force from other agents, also there is attractive force $ \vec{f_{\alpha i}} $ if the agent $ \alpha $ has some relatives or friends in the crowd.

\begin{equation}\label{model}
    \vec{f_{\alpha}} = \vec{f^{0}_{\alpha}} + \vec{f_{\alpha B}} +
    \sum_{\beta \neq \alpha} \vec{f_{\alpha \beta}} +  
    \sum_{i} \vec{f_{\alpha i}} 
\end{equation}

Naturally the work next is to show explicit expression for each force, and we will go through them one at a time explaining their mathematical structure and their role in the model.\\

\subsection{The desired force} %gotta figure out a better name for this part
The first term on the right hand side of equation \eqref{model} describes the implement of agent $ \alpha $'s "willingness" to reach the exit. In this model it is a velocity dependent force 
and is given by:

\begin{equation}\label{relaxtime}
	\vec{f^{0}_{\alpha}}\left( \vec{V_{\alpha}} \right) =
    \frac{1}{\tau}
    \left( V_{\alpha}^{0} \vec{e_{\alpha}} - \vec{V_{\alpha}} \right)
\end{equation}
where $V_{\alpha}^{0}$ is the desired speed, $ \vec{e_{\alpha}} $ is the normal vector pointing to the exit, $\vec{V_{\alpha}}$ is the actual velocity of the agent, and $\tau$ is the relaxation time. \\

From Equation \ref{relaxtime} we get the ideas that:
\begin{itemize}
\item About the notation, any quantity with a "$ ^{0} $ " on the upper right corner is a "desired" quantity.
\item Although $ \vec{f^{0}_{\alpha}} $ is a kind of desired force, it actually exists, because it appears in Equation \ref{model} for calculating the actual acceleration. However, the original article has not pointed out what is the source of  $ \vec{f^{0}_{\alpha}} $.  As the calculation of that force is closely related with the desired velocity $V_{\alpha}^{0}$, then we can think of the desired velocity as a imaginary source if we are not so interested in the physical source. We have considered the static frictional force as a most likely way of achieving the $ \vec{f^{0}_{\alpha}} $, and there are various other ways to fulfil that purpose, for example, by pulling.
\item $ V_{\alpha}^{0} \vec{e_{\alpha}} $ represent the desired velocity, which is needed to calculated in two steps, first to get the magnitude and then the direction. $ \vec{e_{\alpha}} $ is a normal vector pointing to the exit.
\item In Equation \ref{relaxtime}, $\tau$ is used to divide the difference between the desired velocity and the actual velocity. Equation \ref{relaxtime} fits dimensional analysis, because the quantity get from the division is some form of acceleration. 
\item Normally the relaxation time means the time needed to get from one state to another. The article \cite{self-org} suggests $ \tau_{\alpha}\approx 1s $, which shows that it usually takes the agent $ 1s $ to change its velocity.
\end{itemize}
The desired speed at some time $V_{\alpha}^{0}\left( t \right)$ is given by:

\begin{equation}\label{v0eta}
    V_{\alpha}^{0}\left( t \right) = \left[ 1 - \eta_{\alpha} \left( t \right) \right] 
    V_{\alpha}^{0} \left( 0 \right) +
    \eta_{\alpha} \left( t \right)V_{\alpha}^{\text{max}}
\end{equation}
where $V_{\alpha}^{0} \left( 0 \right)$ is the desired speed at $ t=0 $, and $V_{\alpha}^{\text{max}}$ is the maximum desired speed of agent
$\alpha$. \\
$\eta_{\alpha}$ is called the impatience or nervousness of the agent and is given by:

\begin{equation}\label{eta}
	\eta_{\alpha} \left( t \right) =
    1 - \frac{\overline{V}_{\alpha} \left( t \right)}
             {V_{\alpha}^{0} \left( 0 \right)}
\end{equation}
where $\overline{V}_{\alpha}\left( t \right)$ is the average speed in the desired direction.\\\\
As for Equation \ref{v0eta} and Equation \ref{eta}, we have the following considerations:
\begin{itemize}
\item The maximum desired speed $V_{\alpha}^{\text{max}}$ is the speed that agent $\alpha$ will try to get if it is allowed by the 
environment and other agents. 
\item The desired speed $V_{\alpha}^{0} \left( t \right)$ Changes with time, and especially varies with the impatience factor which is also a function of time $ t $.
\item The average speed $\overline{V}_{\alpha} \left( t \right)$ along the desired direction is not specified in the original article, and our understanding of that concept is as in Figure \ref{impatience}, where the projection( the projected vector is called $ \vec{r_{\alpha}^{E}}$ ) of $ \vec{r_{\alpha}} $ onto the desired direction of motion is used to calculate $\overline{V}_{\alpha} \left( t \right)$.

\begin{figure}[ht]
\centering
{\includegraphics[scale=0.35]{Figures/NotationOfAgent2.pdf}} 
\caption{Illustration of the vector $ \vec{r_{\alpha}^{E}}$, which is the projection of $ \vec{r_{\alpha}} $ onto the desired direction of motion.}
\label{impatience}
\end{figure}

Therefore, we want to calculate $\overline{V}_{\alpha} \left( t \right)$ in the following way:
\begin{equation}\label{averagespeed}
   \overline{V}_{\alpha} \left( t \right) = \frac{1}{t} \vec{r_{\alpha}}\cdot \vec{e_{\alpha}} 
\end{equation}
\item From Equation \ref{v0eta} the initial desired speed $V_{\alpha}^{0} \left( 0 \right)$ is used to calculated desired speed at any time $ t $, and if we put $ t=0 $ into the equation, we have
\begin{equation}
    V_{\alpha}^{0}\left( 0 \right) = \left[ 1 - \eta_{\alpha} \left( 0 \right) \right] 
    V_{\alpha}^{0} \left( 0 \right) +
    \eta_{\alpha} \left( 0 \right)V_{\alpha}^{\text{max}}
\end{equation}
where $ \eta_{\alpha} \left( 0 \right) $ is not known.
Also for Equation \ref{eta}, put $ t=0 $ and we get
\begin{equation}
	\eta_{\alpha} \left( 0 \right) =
    1 - \frac{\overline{V}_{\alpha} \left( 0 \right)}
             {V_{\alpha}^{0} \left( 0 \right)}
\end{equation}
although the initial average speed $ \overline{V}_{\alpha} \left( 0 \right) $ is not defined in the original article, we decide to put $ \overline{V}_{\alpha} \left( 0 \right)=0 $, because in that case 
\begin{eqnarray}
	\eta_{\alpha} \left( 0 \right) &=&
    1 - \frac{\overline{V}_{\alpha} \left( 0 \right)}
             {V_{\alpha}^{0} \left( 0 \right)}\\
&=& 1 - \frac{0}{V_{\alpha}^{0} \left( 0 \right)}
= 1
\end{eqnarray}
which makes sense, as when the emergency suddenly happens our agent should feel extremely anxious ($ \eta_{\alpha} \left( 0 \right)=1 $). Then we know the initial desired speed $ V_{\alpha}^{0}\left( 0 \right) $ is
\begin{eqnarray}
    V_{\alpha}^{0}\left( 0 \right) &=& \left[ 1 - \eta_{\alpha} \left( 0 \right) \right] 
    V_{\alpha}^{0} \left( 0 \right) +
    \eta_{\alpha} \left( 0 \right)V_{\alpha}^{\text{max}}\\
&=& \left( 1 - 1 \right)  
    V_{\alpha}^{0} \left( 0 \right) +
    1 V_{\alpha}^{\text{max}}\\
&=& V_{\alpha}^{\text{max}}
\end{eqnarray}
\item Equation (\ref{v0eta}) and Equation (\ref{eta}) contains an intermediate variable $ \eta_{\alpha} \left( t \right) $, 
so in principle we are allowed to eliminate $ \eta_{\alpha} \left( t \right) $ and only show the 
relationship between $ V_{\alpha}^{0}(t) $ and $ \overline{V}_{\alpha} \left( t \right) $. Thus we get:

\begin{equation}\label{vv}
    V_{\alpha}^{0}(t) = \left[ 1 - \frac{V_{\alpha}^{max}}{V_{\alpha}^{0}(0)}\right]\overline{V}_{\alpha} \left( t \right) + V_{\alpha}^{max}
\end{equation}

\begin{figure}[ht]
\centering
{\includegraphics[scale=0.35]{Figures/impatience.pdf}} 
\caption[The impatience factor]{Illustration on the correlation between the function about the desired speed $ V_{\alpha}^{0}(t) $ 
and the average speed in the desired direction of motion $ \overline{V}_{\alpha} \left( t \right) $, and the graph of
$ V_{\alpha}^{0}(t) = \left[ 1 - \frac{V_{\alpha}^{max}}{V_{\alpha}^{0}(0)}\right]\overline{V}_{\alpha} \left( t \right) + V_{\alpha}^{max} $
intersects with the axis at $ \left( 0 , V_{\alpha}^{max} 
\right)  $ and $ \left(V_{\alpha}^{max} 
		\frac{V_{\alpha}^{0} \left( 0 \right) }{V_{\alpha}^{max}-V_{\alpha}^{0} \left(0 \right)} , 0 
\right)  $.
When $\alpha$'s velocity is at maximum the impatience gets low, and vice versa.}
\label{fig:impatience}
\end{figure}

Figure (\ref{fig:impatience}) is a drawing of the graph about those two variables, and the intersection
 of the function line with both axis are:

\begin{equation}
\left( 
	\overline{V_{\alpha}} , V_{\alpha}^{0} \left( t \right)
\right)
=
\left( 
	0 
		, 
	V_{\alpha}^{max} 
\right) 
\text{and} 
\left(
	V_{\alpha}^{max} 
		\frac{V_{\alpha}^{0} \left( 0 \right) }{V_{\alpha}^{max}-V_{\alpha}^{0} \left(0 \right)} 
	, 0 
\right) 
\end{equation}
Normally, the values of the two speeds should have positive values, so the graph is part of a straight line.
Now there is a doubt the range of the value of $ \overline{V}_{\alpha} \left( t \right) $, compared with $ V_{\alpha}^{max} $, 
if we have already 
set $ V_{\alpha}^{max} $ a fixed number for a certain agent. In the case:
\begin{equation}
	V_{\alpha}^{max} 
	\geq 
	V_{\alpha}^{max} 
	\frac{V_{\alpha}^{0}(0)}{V_{\alpha}^{max}-V_{\alpha}^{0}(0)}
\end{equation}
we get the relation:
\begin{equation}
V_{\alpha}^{0}(0)\leq \frac{1}{2} V_{\alpha}^{\text{max}}
\end{equation}
Which contradicts with our earlier conclusion that
\begin{equation}
    V_{\alpha}^{0}\left( 0 \right) = V_{\alpha}^{\text{max}}
\end{equation}
Therefore, the graph should not intersect with $ \overline{V_{\alpha}} $ axis under normal circumstances when the maximum desired velocity is not exceeded.
\item However, from Equation \ref{vv}  and Figure \ref{impatience} we think that $V_{\alpha}^{\text{max}}$ can be exceeded, but only under rare situations. For example, at $ t=0 $, if agent $ \alpha $ moves opposite to the exit because of some extreme large repulsive force, then from Equation \ref{vv} $ V_{\alpha}^{0} \left( 0 \right)  $ is larger than $V_{\alpha}^{\text{max}}$.
\item If there are no repulsive forces at all, the only source of acceleration is from the desired velocity, which modifies the actual velocity to the desired direction and value.  After some time, the actual velocity should reach some constant, which resembles the so called terminal velocity in physics when the acceleration is zero.  When the agent reaches the terminal velocity the actual velocity does not change and it equals the average velocity, so we can write the acceleration from Equation \ref{relaxtime} as

\begin{equation}
\vec{f}_{\alpha} = \vec{f^{0}_{\alpha}}\left( \vec{V_{\alpha}} \right)
\end{equation}
\begin{equation}
\frac{1}{\tau}\left( V_{\alpha}^{0} \vec{e_{\alpha}} - \overline{V}_{\alpha} \left( t\right) \vec{e_{\alpha}}  \right)  = 0
\end{equation}
As $ \tau $ is a constant, despite the direction of the velocity vector we have
\begin{equation}\label{terminal}
	 V_{\alpha}^{0} - \overline{V}_{\alpha} \left( t\right) 
    = 0
\end{equation}

Also we take Equation \ref{vv}, and insert the value of $ V_{\alpha}^{0}(t) $ from Equation \ref{vv} to Equation \ref{terminal}:

\begin{equation}
	\left[ \left( 1 - \frac{V_{\alpha}^{max}}{V_{\alpha}^{0}(0)}\right)\overline{V}_{\alpha} \left( t \right) + V_{\alpha}^{max} \right] - \overline{V}_{\alpha} \left( t\right) 
    = 0
\end{equation}
Solve for $ \overline{V}_{\alpha} \left( t\right) $ we get:
\begin{equation}
\overline{V}_{\alpha} \left( t\right) = V_{\alpha}^{0}(0)
\end{equation}
Which makes a lot of sense because the terminal velocity is the initial desired velocity and equals the maximum desired velocity.

\item The impatience or nervousness factor is active when one calculates the 
force action on agent $\alpha$ from the velocity of the agent.

In the case where $0 \leq \eta_{\alpha} \leq 1$ the expression for 
$V_{\alpha}^{0} \left( t \right)$  makes sense. Here we can see why this term 
is called the impatience of the agent. If the fraction  between the average 
speed in the desired direction and the initial speed is low then $\eta_{\alpha} \approx 1$. 
When the impatience term is close to one $V_{\alpha}^{0} \left( t \right)$ 
is dominated by $V_{\alpha}^{\text{max}}$. That is, if the agent have not 
moved very far in the desired direction compared to the initial speed the 
impatience of the agent will cause the agent's future velocity to be dominated by 
the desired velocity of the agent.

If the agent has been moving in the desired direction with his initial 
speed the entire time then $\eta_{\alpha} = 0$  and 
$V_{\alpha}^{0} \left( t \right)$ will continue to be $V_{\alpha}^{0} \left( 0 \right)$.

In the case where $\eta_{\alpha} \leq 0$ that is the agent has moved further 
in the desired direction then he would have had he been walking with his 
initial speed. The expression for $V_{\alpha}^{0} \left( t \right)$
stats yield strange results. That $\eta_{\alpha} \leq 0$ would imply that:

\begin{equation}\label{n}
    V_{\alpha}^{0} \left( t\right) = \left[ 1 + \eta_{\alpha} \left( t \right) \right] 
    V_{\alpha}^{0} \left( 0 \right) -
    \eta_{\alpha} \left( t \right)V_{\alpha}^{\text{max}}
\end{equation}

And this will yield a negative value for $V_{\alpha}^{0}$ if: 

\begin{equation}
\left[ 1 + \eta_{\alpha} \left( t \right) \right] 
V_{\alpha}^{0} \left( 0 \right) < \eta_{\alpha} \left( t \right)V_{\alpha}^{\text{max}} 
\end{equation}

This is a problem because it is not that far fetched that an agent will be 
forced to exceed his desired velocity.

In the case where $1 \leq \eta_{\alpha}$ it would mean that the agent has moved 
further in the opposite direction than the desired one and this can only happen very 
weird situations.
\end{itemize}

% Lets have a little summation here. What have we learned about the inpatience factor and
% the velocity dependant force. What kind of dynamics does this force yield





\subsection{Repulsion from the walls}
Now the second term on the right hand side of \eqref{model} is a force which arise from interactions with the walls or other obstacles. The forces, caused by the wall or obstacles, is given by:

\begin{equation}\label{wallpotential}
    \vec{f_{\alpha B}} \left( \vec{r_{\alpha}} \right) =
    - \nabla_{\vec{r_{\alpha}}} U_{B}
    \left( \| \vec{r_{\alpha}} - \vec{r_{B}^{\alpha}} \| \right)
\end{equation}
$U_B$ is a repulsive potential and $ \| \vec{r_{\alpha}} - \vec{r_{B}^{\alpha}} \|$ is the distance 
from the position of agent $\alpha$ to the nearest point $ \vec{r_{B}^{\alpha}}  $ of the wall and shown in figure \ref{NotationOfWall}.

\begin{figure}[ht]
\centering
{\includegraphics[scale=0.35]{Figures/NotationOfWall.pdf}} 
\caption[Notation of the interaction between an agent and a wall]{The illustration shows the mathematical notation for the interaction with walls used. The circle is pedestrian $\alpha$ with radius $r_{\alpha}$, $\vec{r_{\alpha}}$ is the position vector for $\alpha$, the grey box on the top is the wall, $\vec{r_{B}^{\alpha}}$ is the position vector for the closest part of the wall to $\alpha$, $\left( \| \vec{r_{\alpha}} - \vec{r_{B}^{\alpha}} \| \right)$ is the smallest distance from $\alpha$ to the wall and $\vec{V_{\alpha}}$ is the velocity vector for $\alpha$.}
\label{NotationOfWall}
\end{figure}

\begin{itemize}
\item  $U_B$ only depends on the distance $ \| \vec{r_{\alpha}} - \vec{r_{B}^{\alpha}} \|$, so the gradient of $V_B$ tells us in which direction does this distance change the most. It is obvious that the changes is largest if the agent takes a step directly towards or away from the wall, which means that the agent will always be pushed directly away from the wall.

\item The next job is to find an explicit expression for $ \| \vec{r_{\alpha}} - \vec{r_{B}^{\alpha}} \|$\\
To start of with we need to find point on the wall that is perpendicular to $\alpha$ as it will be the nearest.  
If we define the wall as a vector, $\vec{W}$ going from a point in space $w_1$ to another point $w_2$, then we can find the projection of $\alpha$ onto the wall and this projection will be $\vec{r_{B}^{\alpha}}$.
The equation for the projection is
\begin{equation}\label{wall}
\vec{r_{B}^{\alpha}}=\frac{\vec{r_{\alpha}}\cdot \vec{W}}{\| \vec{W} \|^2}\vec{W}
\end{equation}
With this we now have the two points we need to calculate $ \| \vec{r_{\alpha}} - \vec{r_{B}^{\alpha}} \|$.


\item With the explicit expression for $ \| \vec{r_{\alpha}} - \vec{r_{B}^{\alpha}} \| $, we are able to calculate $ \vec{f_{\alpha B}} \left( \vec{r_{\alpha}} \right) $ from Equation \ref{wallpotential}, if the expression for the potential function $ V_{B}
    \left( \| \vec{r_{\alpha}} - \vec{r_{B}^{\alpha}} \| \right) $ is given.\\
In some of the other articles [ ] made by the same outhers as the article which makes the basis for the model in this repport, the repulsive potential from the wall is given as
\begin{equation}
U_{B} \left( \| \vec{r_{\alpha}} - \vec{r_{B}^{\alpha}} \| \right) =
U^0_{\alpha B} e^{- \| \vec{r_{\alpha}} - \vec{r_{B}^{\alpha}} \| / r_{\alpha} }
\end{equation}
where $U^0_{\alpha B}$ is a constant and $r_{\alpha}$ is the radius of a pedestrian $\alpha$. \\

In that case, the repulsive force on agent $ \alpha $ from the wall is:
\begin{equation}
    \vec{f_{\alpha B}} \left( \vec{r_{\alpha}} \right) =
    - \nabla_{\vec{r_{\alpha}}} U_{B}
    \left( \| \vec{r_{\alpha}} - \vec{r_{B}^{\alpha}} \| \right)\\
=-\left( \frac{\partial}{\partial x_{\alpha}}U_{B}( \| \vec{r_{\alpha}} - \vec{r_{B}^{\alpha}} \|), \frac{\partial}{\partial y_{\alpha}}U_{B}( \| \vec{r_{\alpha}} - \vec{r_{B}^{\alpha}} \|)\right) \\
\end{equation}
Calculating the derivatives we use that $\| \vec{r_{\alpha}} - \vec{r_{B}^{\alpha}} \|= \sqrt{(x_{\alpha}-x^{\alpha}_{B})^2+(y_{\alpha}-y^{\alpha}_B)^2}$
So if we insert the expression for $U_{B}$ and for $\| \vec{r_{\alpha}} - \vec{r_{B}^{\alpha}} \|$ we get

\begin{equation}
    \vec{f_{\alpha B}} \left( \vec{r_{\alpha}} \right) 
=-\left( \frac{\partial}{\partial x_{\alpha}}U^0_{\alpha B} e^{-\sqrt{(x_{\alpha}-x^{\alpha}_{B})^2+(y_{\alpha}-y^{\alpha}_B)^2} / r_{\alpha} }, \frac{\partial}{\partial y_{\alpha}}U^0_{\alpha B} e^{- \sqrt{(x_{\alpha}-x^{\alpha}_{B})^2+(y_{\alpha}-y^{\alpha}_B)^2} / r_{\alpha} } \right)
\end{equation}
When we differentiate this, we will put $u=-\sqrt{(x_{\alpha}-x^{\alpha}_{B})^2+(y_{\alpha}-y^{\alpha}_B)^2} / r_{\alpha} $ and then use the well known result, from use of the chain rule when using differentation on an exponential function:
\begin{equation}
\frac{\partial d}{\partial x}Ae^{u}=\frac{\partial u}{\partial x}Ae^{u}
\end{equation}
This gives
\begin{equation}
    \vec{f_{\alpha B}} \left( \vec{r_{\alpha}} \right) 
=-\left( \frac{\partial -\sqrt{(x_{\alpha}-x^{\alpha}_{B})^2+(y_{\alpha}-y^{\alpha}_B)^2} / r_{\alpha}}{\partial x_{\alpha}}U^0_{\alpha B} e^{-\sqrt{(x_{\alpha}-x^{\alpha}_{B})^2+(y_{\alpha}-y^{\alpha}_B)^2} / r_{\alpha} }, \frac{\partial -\sqrt{(x_{\alpha}-x^{\alpha}_{B})^2+(y_{\alpha}-y^{\alpha}_B)^2} / r_{\alpha}}{\partial y_{\alpha}}U^0_{\alpha B} e^{- \sqrt{(x_{\alpha}-x^{\alpha}_{B})^2+(y_{\alpha}-y^{\alpha}_B)^2} / r_{\alpha} } \right)
\end{equation}
Differentation the square root we get 
\begin{equation}
 \frac{\partial -\sqrt{(x_{\alpha}-x^{\alpha}_{B})^2+(y_{\alpha}-y^{\alpha}_B)^2} / r_{\alpha}}{\partial x_{\alpha}} =-\frac{1}{r_{\alpha}}\frac{(x_{\alpha}-x_{B}^{\alpha})}{\sqrt{(x_{\alpha}-x^{\alpha}_{B})^2+(y_{\alpha}-y^{\alpha}_B)^2} }
\end{equation}
If we do this and afterwards substitute bakc with $\| \vec{r_{\alpha}} - \vec{r_{B}^{\alpha}} \|= \sqrt{(x_{\alpha}-x^{\alpha}_{B})^2+(y_{\alpha}-y^{\alpha}_B)^2}$ we get the final result:
\begin{equation}
    \vec{f_{\alpha B}} \left( \vec{r_{\alpha}} \right) 
=-\left(\left(U^0_{\alpha B}\frac{1}{r_{\alpha}}\frac{e^{- \| \vec{r_{\alpha}} - \vec{r_{B}^{\alpha}} \| / r_{\alpha} } (x_{\alpha}-x_B^{\alpha})}{\| \vec{r_{\alpha}} - \vec{r_{B}^{\alpha}} \| }\right),\left(U^0_{\alpha B}\frac{1}{r_{\alpha}}\frac{e^{- \| \vec{r_{\alpha}} - \vec{r_{B}^{\alpha}} \| / r_{\alpha} } (y_{\alpha}-y_B^{\alpha})}{\| \vec{r_{\alpha}} - \vec{r_{B}^{\alpha}} \| }\right)\right) \\
\end{equation}

\item The exponential function will always lay between 0 and 1:
\begin{equation}
0 < e^{ -\| \vec{r_{\alpha}} - \vec{r_{B}^{\alpha}} \| /r_\alpha} < 1
\end{equation}
Which leads to
\begin{equation}
0< U_{B} \left( \| \vec{r_{\alpha}} - \vec{r_{B}^{\alpha}} \| \right) < U^0_{\alpha B}
\end{equation}
We can see that this force act in the following way:\\
$\vec{f_{\alpha B}}$ tends to 0 as the distance $ \| \vec{r_{\alpha}} - \vec{r_{B}^{\alpha}} \|$ gets large, meaning that a pedestrian in a reasonably distance from the wall will feel a diminishing force. $\vec{f_{\alpha B}}$ tends to $V^0_{\alpha B}$ as the distance $ \| \vec{r_{\alpha}} - \vec{r_{B}^{\alpha}} \|$ tends to $0$ the pedestrian will be pushed, with some force depending on $V^0_{\alpha B}$, away from the wall. 
The negative of the force means that when the potential between wall and pedestrian rises so will the force, but in the opposite direction meaning that the pedestrian will be pushed away. This can be understood as if the pedestrian is trying to avoid the wall as it is expected from real life situations.
 [Helbing and Molnár, 1995]. %real references, please.
\end{itemize}


\subsection{Repulsion from other agents}
The third term on the right hand side of \eqref{model} is a summation of all the 
force between agent $\alpha$ and agent $\beta$. It is a function of the position vector and the velocity of 
both agents, and it is given by:

\begin{equation}
    \sum_{\beta \left( \neq \alpha \right)}
        \vec{f_{\alpha \beta }}\left( t \right) =
        A_{\alpha}^{1} exp \left(
            \frac{ r_{\alpha \beta} - d_{\alpha \beta }}
                 {B_{\alpha}^1}
        \right)
    \vec{\eta_{\alpha \beta}} \cdot
    \left(
        \lambda_{\alpha} + \left(
            1 - \lambda_{\alpha}
        \right)
		\frac{1+\cos{\phi}}{2}
    \right) +
    A_{\alpha}^{2} exp\left(
        \frac{r_{\alpha \beta} - d_{\alpha \beta}}
             {B_{\alpha}^{2}}
    \right)
    \vec{\eta_{\alpha \beta}}
    \label{agentinteraction}
\end{equation}

\begin{figure}[ht]
    \centering
    {\includegraphics[scale=0.35]{Figures/NotationOfInteraction.pdf}} 
    \caption[Notation of the interaction between two agents]{Illustration of the notation for the interaction between agents.
	     An addition and difference to \ref{NotationOfWall} is that the wall has been replaced by pedestrian $\beta$.
	     $\eta_{\alpha \beta}$ is the normal vector pointing from $\alpha$ to $\beta$, and $\phi$ is the angle between $\alpha$'s 
	     velocity vector and $\beta$'s center of mass.}
    \label{NotationOfInteraction}
\end{figure}

Here $A_{\alpha}^{1}$, $A_{\alpha}^{2}$, $B_{\alpha}^{1}$, $B_{\alpha}^{2}$ 
and $\lambda_{\alpha}$ are all constants that can differ for each agent. 
$r_{\alpha \beta}$ is the sum of the radii of $\alpha$ and $\beta$ that is 
$r_{\alpha \beta} = r_{\alpha} + r_{\beta}$. $d_{\alpha \beta}$ is the 
distance from the center of mass of agent $\alpha$ and the center of mass of 
agent $\beta$ and is therefore given by $d_{\alpha \beta} = 
\|\vec{r_{\alpha}}\left( t \right) - \vec{r_{\beta}}\left( t \right) \|$.
$\eta_{\alpha \beta}$ is the normal vector pointing from $\alpha$ to $\beta$ 
and it is given by:

\begin{equation}
    \eta_{\alpha \beta} =
        \frac{\vec{r_{\alpha}}(t) - \vec{r_{\beta}}(t)}
             {\|\vec{r_{\alpha}}(t) - \vec{r_{\beta}}(t) \|}
\end{equation}

the angle $\phi$ in \eqref{agentinteraction} is the angle between the normal 
vector pointing from agent $\beta$ to $\alpha$ and the direction in which 
agent $\alpha$ is moving. Cosine to the angle is 

\begin{equation}
\cos \left( \phi \right)
	\left( t \right) 
		= 
	- \vec{\eta_{\alpha \beta}}
		\left( t \right) 
	\cdot 
\vec{e_{\alpha}}\left( t \right)
\end{equation}

Equation \eqref{agentinteraction} is divided into two terms. The first term on 
the right hand side reflects the agents tendency to stay at a certain distance 
from other agents. This part of the force is called the private sphere because 
the agent prefers to have some free space around him if possible. The radius 
of the private sphere can differ from agent to agent. The constant 
$A_{\alpha}^{1}$, $B_{\alpha}^{1}$ and $\lambda_{\alpha}$ control the nature 
of the private sphere $A_{\alpha}^1$ and $B_{\alpha}^1$ control the strength 
and range of the interaction respectively. $\lambda_{\alpha}$ is there to take 
into account a persons tendency to focus on things happening in front of him 
rather than behind him.	% we should make a drawing of this.

The second term of equation \eqref{agentinteraction} deals with physical interaction.
In the situation where the density of the crowd is high the agents will have be closer
to each other and the social sphere is undermined. % go more into detail here 
So if we look away from the social sphere for a minute and concentrate on the physical
interaction we will see that if we omit the social sphere the calculation will be reduced to
to:

\begin{equation}\label{re2}
\overrightarrow{f_{\alpha\beta}}(t) = A_{\alpha}^{2} exp\left[ \frac{r_{\alpha\beta} - d_{\alpha}\beta}{B_{\alpha}^{2}}\right]  \overrightarrow{n_{\alpha\beta}}
\end{equation}

Taking the norms of both sides of Equation (\ref{re2}), we can draw the relation between the value of $\overrightarrow{f_{\alpha\beta}}(t)$ and $ d_{\alpha\beta} $, as shown in Figure 
(\ref{fig:physicalinteraction1}).\\

\begin{figure}
    \centering
    {\includegraphics[scale=0.45]{Figures/physicalinteraction.pdf}} 
    \caption[Psysical interaction]{Illustration of the function about the interaction force 
        $f_{\alpha\beta}(t)$ and the distance between two agents
        $d_{\alpha \beta}$. It follows that the smaller the distance between two agents, the greater the interaction force is. }
    \label{fig:physicalinteraction1}
\end{figure}

There is one intersection of the graph and the  axis at:

\begin{equation}
	\left( d_{\alpha \beta} , \| \vec{f_{\alpha \beta}} \left( t \right) \| \right)
 =
	\left( 0 , A_{\alpha}^{2} exp\left( \frac{r_{\alpha\beta} }{B_{\alpha}^{2}}\right)  \right) 
\end{equation}

If put into the constants, we will be able to get a maximum value of $ f_{\alpha\beta}(t) $, 
since the distance between agents cannot be negative. Here we set $ A_{\alpha}^{2} = 3 m/s^{2} $, 
$ r_{\alpha\beta} = 0.6 m $, and $ B_{\alpha}^{2} = 0.2 m $, so 
$ f_{\alpha\beta}(t)^{max} \doteq 60 m/s^{2} $, which is about six times the gravitational 
acceleration and represents a rather large force between agents (as large as six person's weight).

%TODO: remember to finish this section

% TODO: again lets  have a little summation here. What kinds of dynamics does the
% social interaction part of the model yield. 


\subsection{Repulsion from other agents}
The third term on the right hand side of \eqref{model} is a summation of all the 
force between agent $\alpha$ and agent $\beta$. The equation we use is taken from a newer article \cite{ABconstant}. We did this because the original article contains two constants that is not given in the articel and in a mail we were adviced, by the authors, to use the values and equation from the new article. 

The function for the repulsion between pedestrians depends on the position vector and the velocity of 
both agents, and it is given by:

\begin{equation}
        \vec{f_{\alpha \beta }}\left( t \right) = w\left(\phi_{\alpha \beta}\right)\vec{g}\left(d_{\alpha \beta}(t)\right)
    \label{eq:agentinteraction}
\end{equation}

\begin{itemize}
\item $w\left(\phi_{\alpha \beta}\right)$ is a function of the angle between two pedestrians. It is given as: 

\begin{equation}
    w\left(\phi_{\alpha \beta}\right)=
    \vec{\eta_{\alpha \beta}} \cdot
    \left(
        \lambda_{\alpha} + \left(
            1 - \lambda_{\alpha}
        \right)
		\frac{1+\cos{\phi}}{2}
    \right) 
    \label{angleAB}
\end{equation}

the angle $\phi_{\alpha \beta}$ is the angle between the normal 
vector pointing from agent $\beta$ to $\alpha$ and the direction in which 
agent $\alpha$ is moving. Cosine to the angle is 

\begin{equation}
\cos \left( \phi \right)
	\left( t \right) 
		= 
	- \vec{\eta_{\alpha \beta}}
		\left( t \right) 
	\cdot 
\vec{e_{\alpha}}\left( t \right)
\end{equation}

$\lambda_{\alpha}$ is governing a persons tendency to focus on things happening in front of him 
rather than behind him. It will have a value  $0\leq \lambda_{\alpha}\leq 1$

\item A value of $\lambda_{\alpha}=1$ means that the force won't depent on the angle. Thus $\alpha$ will react the same to $\beta$ no matter if $\beta$ is in the front or comes from the side or back. A value of $0$ will on the other hand give the maximum angle depence. We find that $0\leq w\left(\phi_{\alpha \beta}\right)\leq1$ when $-1 \leq \cos \left( \phi \right) \left( t \right) \leq 1$. From this we see that $\alpha$ wont be affected at all if $\beta$ is coming from behind and the force will be maximum when $\beta$ comes directly in the front. It should be noted that in general one will find that the maximum of $w\left(\phi_{\alpha \beta}\right)$ always will be $1$ and the minimum will be equal to $\lambda_{\alpha}$.   

	% we should make a drawing of this.

\item The force, without the effect of $w\left(\phi_{\alpha \beta}\right)$, is given as the second term, in the right hand side of equation \ref{agentinteraction}. The function is:  
\begin{equation}
	\vec{g} 
	\left(
	d_{\alpha \beta}
	\right)
	=
	 A_{\alpha} e^{ \left(\frac{ r_{\alpha \beta} - d_{\alpha \beta}}{B_{\alpha}}\right)}
	\vec{n}_{\alpha \beta}
	        \label{re}	
\end{equation}

Here $A_{\alpha}$ and $B_{\alpha}$ are constants that can differ for each agent. 
$r_{\alpha \beta}$ is the sum of the radii of $\alpha$ and $\beta$ that is 
$r_{\alpha \beta} = r_{\alpha} + r_{\beta}$. $d_{\alpha \beta}$ is the 
distance from the center of mass of agent $\alpha$ and the center of mass of 
agent $\beta$ and is therefore given by $d_{\alpha \beta} = 
\|\vec{r_{\alpha}}\left( t \right) - \vec{r_{\beta}}\left( t \right) \|$.
$\eta_{\alpha \beta}$ is the normal vector pointing from $\alpha$ to $\beta$ 
and it is given by:

\begin{equation}
    \eta_{\alpha \beta} =
        \frac{\vec{r_{\alpha}}(t) - \vec{r_{\beta}}(t)}
             {\|\vec{r_{\alpha}}(t) - \vec{r_{\beta}}(t) \|}
\end{equation}

\begin{figure}[ht]
    \centering
    {\includegraphics[scale=0.35]{Figures/NotationOfInteraction.pdf}} 
    \caption[Notation of the interaction between two agents]{Illustration of the notation for the interaction between agents.
	     An addition and difference to \ref{NotationOfWall} is that the wall has been replaced by pedestrian $\beta$.
	     $\eta_{\alpha \beta}$ is the normal vector pointing from $\alpha$ to $\beta$, and $\phi$ is the angle between $\alpha$'s 
	     velocity vector and $\beta$'s center of mass.}
    \label{fig:NotationOfInteraction}
\end{figure}


%Equation \ref{agentinteraction} is given by two terms. 
%The first term determines how much the the angle between the agents is and how much this angle should affect the force. 
%The constant $\lambda$ is the one controling the importance of the angle.
%The second term reflects the pedestrians tendency to stay at a certain distance 
%from other agents. The constants $A_{\alpha}$, $B_{\alpha}$ is the strength 
%and range of the interaction respectively and controls how big the force will be at a certain distance.  




\item Taking the norms of both sides of Equation (\ref{re}), we can draw the relation between the value of $\overrightarrow{f_{\alpha\beta}}(t)$ and $ d_{\alpha\beta} $, as shown in Figure 
(\ref{fig:physicalinteraction2}).\\

\begin{figure}
    \centering
    {\includegraphics[scale=0.45]{Figures/physicalinteraction.pdf}} 
    \caption[Psysical interaction]{Illustration of the function about the interaction force 
        $f_{\alpha\beta}(t)$ and the distance between two agents
        $d_{\alpha \beta}$. It follows that the smaller the distance between two agents, the greater the interaction force is. }
    \label{fig:physicalinteraction2}
\end{figure}

There is one intersection of the graph and the  axis at:

\begin{equation}
	\left( d_{\alpha \beta} , \| \vec{f_{\alpha \beta}} \left( t \right) \| \right)
 =
	\left( 0 , A_{\alpha} exp\left( \frac{r_{\alpha\beta} }{B_{\alpha}}\right)  \right) 
\end{equation}

\item If we put in values for the constants, we will be able to get a maximum value of $ f_{\alpha\beta}(t) $, 
since the distance between agents cannot be negative. Here we take values from \cite{ABconstant} $ A_{\alpha} = 0.42 m/s^{2} $, 
$ r_{\alpha\beta} = 0.6 m $, and $ B_{\alpha} = 1.65 m $, so 
$ f_{\alpha\beta}(t)^{max} \doteq 0.64 m/s^{2} $.

\end{itemize}

%TODO: remember to finish this section

%TODO: again lets  have a little summation here. What kinds of dynamics does the
% social interaction part of the model yield.

\subsection{The attractive forces between some agents}
The fourth and last term in \eqref{model} represents the force from attraction 
in the room. Attractions can be either be either interesting sculptures or 
sights or familiar persons the agent prefer to be close to, such as friends 
and family. The mathematical structure of this force is the same as the force 
from other agents, however it is opposite in algebraic sign and has different 
constants. 

To get an overview of how the model is put together look a figure \ref{overview}
with the aid of table \ref{tableofconstandvar}

\begin{figure}[hb] %with some more comments i think that this figure could serve as a summation of the entire section
    \centering
    {\includegraphics[scale=0.45]{Figures/overview.pdf}} 
    \caption[Overview of the model]{Illustration of an overview of how the model is put together. The different equations and their notation is written to give the 
	     reader an overview of how the model looks like.}
    \label{overview}
\end{figure}