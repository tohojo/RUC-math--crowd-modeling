\section{The model}
\label{sec:the-model}
In this section we will go through the model presented in \cite{self-org} in
detail. As mentioned in the introduction, social force models are agent-based,
that is they describe the system by looking at the behaviour of each agent or,
in our case, pedestrian in the system.


In this model the behaviour of each pedestrian is defined by a series of social
forces. We will describe three main forces, that each represent a tendency of
the pedestrians. These forces are \emph{the desired movement force},
\emph{the interaction force} between pedestrians and \emph{the repulsive
force}  from the walls. The model also contains an attractive force, allowing
e.g. a group of pedestrians to stick together when moving around, and a
fluctuation term, adding a stochastic quality to the model. Both of these forces,
however, are only mentioned in the article, and are not really used. We have
written to the authors of the article asking about this, and they have replied
that these parts of the model have not been used in practice, but that they
might be useful to include at a later point in time.  Because of this, we will
exclude these parts from the explanation of the model.

In the following, we will go through the three main forces, explaining how
they are calculated. Since the notation used in the article is ambiguous and
confusing, we introduce our own notation in an attempt to make it clearer. We
will not refer to the original notation, but only use our own.

In the model, a pedestrian is represented by a circle with a centre, a radius
and a current velocity. Walls are represented as line segments. The desired
direction is represented by assigning to each pedestrian a target to move
towards. Various other parameters are assigned to each pedestrian; these will
be explained when relevant throughout the description. The notation for
pedestrians and the main forces acting on a pedestrian are illustrated in
figure~\ref{pedestrian-notation}.

\begin{figure}[ht]
    \centering
    \subfloat[Properties describing a pedestrian.]{
        \includegraphics[width=0.40\textwidth]{Figures/NotationOfPedestrian.pdf}
        \label{subfig:notation}
    }
    \subfloat[Forces acting on a pedestrian.]{
        \includegraphics[width=0.45\textwidth]{Figures/ForceModel.pdf}
        \label{subfig:forces}
}
    \caption[Notation for pedestrians]{
    \subref{subfig:notation} A pedestrian is described by a position
    vector ($ \overrightarrow{p_{\alpha}} $), a velocity vector ($
    \overrightarrow{V_{\alpha}} $), a vector pointing towards the target
    ($\overrightarrow{e_{\alpha}}$)  and a radius ($ R_{\alpha} $). The 
    velocity is determined from the forces action upon the pedestrian.\\
    %
    \subref{subfig:forces} The forces acting on a pedestrian $\alpha$ is the
    interaction force from pedestrian $\beta$
    ($\overrightarrow{f_{\alpha\beta}}$), the repulsive force from wall 
    $\gamma$
    ($\overrightarrow{f_{\alpha \gamma}}$) and the force towards the target 
    ($\overrightarrow{f^{d}_{\alpha}}$).}

    \label{pedestrian-notation}
\end{figure}

The resulting force for pedestrian $\alpha$, $\overrightarrow{f_{\alpha}}$ is
the sum of the three main forces:

\begin{equation}\label{model}
    \overrightarrow{f_{\alpha}} = \overrightarrow{f^{d}_{\alpha}} +
    \sum_{\beta \neq \alpha} \overrightarrow{f_{\alpha \beta}} +
    \sum_{\gamma} \overrightarrow{f_{\alpha \gamma}}
\end{equation}

Here $\overrightarrow{f_{\alpha}^{d}}$ is the desired movement force,
$\overrightarrow{f_{\alpha \beta}}$ is the interaction force from pedestrian
$\beta$ and $\overrightarrow{f_{\alpha \gamma}}$ is the repulsive force from
wall $\gamma$. The main forces are summarised in table~\ref{tbl:main-forces}
and will be described in detail in the following sections.

\begin{table}[h]
    \centering
    \begin{tabular}{l l}
        \toprule
        \multicolumn{2}{c}{\textsf{Main forces}}\\
        $\overrightarrow{f_{\alpha}^{d}}$ & Desired movement force\\
        $\overrightarrow{f_{\alpha \beta}}$ & Interaction force from pedestrian
        $\beta$\\
        $\overrightarrow{f_{\alpha \gamma}}$ & Repulsive force from wall
        $\gamma$\\
        \midrule
        \multicolumn{2}{c}{\textsf{Base notation}}\\
        $\overrightarrow{p_{\alpha}}$ & Position vector of pedestrian
        $\alpha$\\
        $\overrightarrow{V_{\alpha}}$ & Velocity of pedestrian $\alpha$ \\
        \addlinespace[0.3em]
        $R_\alpha$ & Radius of pedestrian $\alpha$\\
        \bottomrule
    \end{tabular}
    \caption{Summary of main forces and base notation.}
    \label{tbl:main-forces}
\end{table}

\subsection{The desired movement force}
\label{sec:desired-force}
The \emph{desired movement force} represents the pedestrians' desire to move towards its
target. The idea is that a pedestrian, if unhindered, will move towards its
target at an initial desired speed that is given as a model parameter. The
desired movement force is a velocity dependent force given by:

\begin{equation}\label{eqn:desired-force}
	\overrightarrow{f^{d}_{\alpha}} (t) =
    \frac{1}{\tau}
    \left( V_{\alpha}^{d}(t) \overrightarrow{e_{\alpha}} -
    \overrightarrow{V_{\alpha}}(t) \right)
\end{equation}

Here $V_{\alpha}^{d}(t)$ is the desired speed at time $t$,
$\overrightarrow{e_{\alpha}}$ is the unit vector pointing towards the
pedestrian's target and  $\overrightarrow{V_{\alpha}}(t)$ is the actual
velocity of the pedestrian at time $t$. It is not clear from the article what
the initial velocity of the pedestrians are (i.e.
$\overrightarrow{V_\alpha}(0)$), so we have assumed it to be zero.


$\tau$ is the \emph{relaxation time} and determines how fast a pedestrian
returns to its desired velocity after having been walking slower because of
obstacles etc. The relaxation time is a model parameter that in principle can vary
for each pedestrian. However, from \cite{self-org} it is clear that in
practice, $\tau$ is the same for all pedestrians in the simulation.

$V_{\alpha}^{d}(t)$ is the desired speed of the pedestrian. The desired speed
can vary over time and is given by:

\begin{equation}\label{eqn:desired-speed}
    V_{\alpha}^{d}(t) = \left( 1 - \eta_{\alpha}(t) \right)
    V_{\alpha}^{Id} +
    \eta_{\alpha}(t) V_{\alpha}^{\text{max}}
\end{equation}

Here $V_{\alpha}^{Id}$ is the \emph{initial desired speed} (a model
parameter), and $V_{\alpha}^{\text{max}}$ is the \emph{maximum desired speed}
of pedestrian $\alpha$. The maximum desired speed is given as a parameter and
is calculated as a function of the initial desired speed by multiplying with a
factor larger than one. This speed is the maximum speed the pedestrian will
try to accelerate to when compensating for slowing down because of obstacles.

The value $\eta_{\alpha}(t)$ is called the \emph{impatience factor} of the
pedestrian and is given by:

\begin{equation}\label{eqn:impatience}
	\eta_{\alpha}(t) =
    1 - \frac{\langle V_{\alpha}(t)\rangle}{V^{Id}_{\alpha}}
\end{equation}

where $\langle V_{\alpha}(t) \rangle$ is the average speed in the desired
direction for all times $0\dots t$. It is not clear from the article exactly
how this average speed is calculated. We have defined it by projecting the
vector pointing from the pedestrian's initial position $p^i_\alpha$ to the 
current   position $p_\alpha$,  onto the vector pointing from $p^i_\alpha$ to 
the pedestrian's target. The length of this projection is divided by the time 
to yield the average speed. An illustration of this interpretation is given in 
figure~\ref{impatience}.

\begin{figure}[ht]
    \centering
    {\includegraphics[width=0.5\textwidth]{Figures/impatience.pdf}}
    \caption[Our interpretation of the average velocity]{Our interpretation of
    the average velocity. The vector pointing from the pedestrian's initial
    position $p^i_\alpha$ to the current position $p_\alpha$, is projected 
    onto the vector pointing from $p^i_\alpha$ to the pedestrian's target. The 
    length of this projection is divided by the time to yield the average 
    speed towards the target.}
    \label{impatience}
\end{figure}

$V^d_\alpha(t)$ is undefined at $t=0$, so that would result in a
division by zero. We complete the definition by defining 
$V^d_\alpha(0)=V^{Id}_\alpha$. This means that $V^d_\alpha(t)$
is defined as follows:

\begin{equation}\label{eqn:cond-define}
    V_{\alpha}^{d} (t) = \left\{
    \begin{array}{l l}
        V_\alpha^{Id} & \text{if $t=0$}\\
        \left( 1 - \eta_{\alpha}(t) \right)
        V^{Id}_{\alpha} +
        \eta_{\alpha}(t) V_{\alpha}^{\text{max}}
        & \text{if $t > 0$}\\
    \end{array} \right.
\end{equation}

Analysing equations~\eqref{eqn:desired-speed} and \eqref{eqn:impatience}, it
is clear that as the average speed approaches $V^{Id}_\alpha$ so does the
desired speed, since $\eta_\alpha(t)\rightarrow0$ for $\langle V_\alpha(t)
\rangle \rightarrow V^{Id}_\alpha$ which makes $V^d_\alpha(t)$ be dominated by
$V^{Id}_\alpha$.

If $\eta_\alpha(t)$ is considered as a function of the average velocity
instead of the time, it is monotonously decreasing. This means that
$V^d_\alpha(t)$ considered as a function of the average velocity is
monotonously decreasing if $V^{\text{max}}_\alpha > V^{Id}_\alpha$,
monotonously increasing if $V^{\text{max}}_\alpha < V^{Id}_\alpha$, and
constant if $V^{\text{max}}_\alpha = V^{Id}_\alpha$. This function also has
the property that $V^d_\alpha(t)=V^{Id}_\alpha$ when $\langle V_\alpha(t)
\rangle = V^{Id}_\alpha$.

Since $V^{\text{max}}_\alpha$ is set to be larger than $V^{Id}_\alpha$, this
means that as the average speed goes up, the desired speed goes down.  When
the desired speed is lower than the actual speed, this will result in a
negative desired movement force due to~\eqref{eqn:desired-force}, and vice
versa. The end result is that the desired movement force will act as a
stabilising force, that accelerates the pedestrian when it is moving slower
than its desired speed, and decelerate it when it is moving faster, but
allowing for movement of up to the maximum desired speed to ``make up'' for
lost time if the pedestrian has been slowed down by the other forces acting on
it.

The components of the desired movement force are summarised in
table~\ref{tbl:desired-force}.

\begin{table}[h]
    \centering
    \begin{tabular}{l l}
        \toprule
        \multicolumn{2}{c}{\textsf{Parameters of the desired movement force}}\\
        $\overrightarrow{V_{\alpha}}(t)$ & Velocity of pedestrian $\alpha$
        at time $t$\\
        $V_{\alpha}^{d}(t)$ & Desired speed of pedestrian $\alpha$ at time
        $t$\\
        $V_{\alpha}^{Id}$ & Initial desired speed of pedestrian $\alpha$ \\
        $\langle V_{\alpha}(t) \rangle$ & Average speed of pedestrian
        $\alpha$ \\
        $\tau$& Relaxation time \\
        \bottomrule
    \end{tabular}
    \caption{Summary of parameters of the desired movement force}
    \label{tbl:desired-force}
\end{table}

\subsection{Interaction force between pedestrians}
\label{seq:interaction-pedestrians}
The \emph{interaction force} between pedestrians is a force between each pair
of pedestrians $\alpha$ and $\beta$. This force is calculated for $\alpha$ as
a sum of the interaction forces for all other pedestrians. The interaction
force is repulsive in nature, and prevents the pedestrians from overlapping or
walking through each other.

The variant of this force that we present is slightly different from the one
in \cite{self-org}.  The force from the article contained too parts with
different constants but otherwise identical; however, no explanation was given
for how to obtain one set of constants. We have written the authors of the
article, and they suggested we use the values from a newer article,
\cite{ABconstant}. The force presented in this article corresponds to
collapsing the two parts given in the original article into one, and adjusting
the constants accordingly.

The function for the interaction force between pedestrians depends on the
position and velocity of both pedestrians, and is given by:

\begin{equation}
    \overrightarrow{f_{\alpha \beta }}(t) =
    w\left(\theta_{\alpha \beta}\right)
    \overrightarrow{g}\left(d_{\alpha \beta}(t)\right)
    \label{eq:pedestrianinteraction}
\end{equation}

Here $\theta_{\alpha \beta}$ is the angle between the movement direction of
$\alpha$ and the vector pointing from $\alpha$ to $\beta$ and $d_{\alpha
\beta}$ is the distance between the centres of the pedestrians.

$ w(\theta_{\alpha \beta})$ is given as:

\begin{equation}
    w\left(\theta_{\alpha \beta}\right)=
    \left(
        \lambda + \left(
            1 - \lambda
        \right)
		\frac{1+\cos{\theta}}{2}
    \right)
    \label{angleAB}
\end{equation}

Where $\lambda$ is a parameter of the model between zero and one. This means
that $w$ is a weight between zero and one that modifies the force given by
$\overrightarrow{g}\left(d_{\alpha \beta}(t)\right)$ by an angular component.
When $\lambda=1$, so is $w$, and no angular modification is made. When
$\lambda<1$, the force is modified by a weight determined by the angle, giving
higher weights to angles closer to $0$ and lower weights to angles close to
$\pi$. This corresponds to the pedestrian paying more attention to (and thus
interacting more strongly with) other pedestrians in front of it, than behind
it. The $\lambda$ parameter determines how pronounced this anisotropic
property of the model is.

% TODO: we should make a drawing of this.

The interaction force, $\overrightarrow{g}\left(d_{\alpha \beta} (t)\right)$, is
given by:

\begin{equation}
    \overrightarrow{g}
    \left(
        d_{\alpha \beta}
    \right)
    =
    A e^{ \left(
        \frac{ R_\alpha + R_\beta - d_{\alpha \beta}}
             {B}
    \right)}
    \overrightarrow{u_{\beta \alpha}}
    \label{re}	
\end{equation}

Here $\overrightarrow{u_{\beta \alpha}}$ is the unit vector pointing from
$\beta$ to $\alpha$ and $A$ and $B$ are model parameters. In the original
article it is specified that they may vary between pedestrians. However, in
practice they have a single value, so we treat them as such. The notation for
the interaction force is summarised in
figure~\ref{subfig:notation-interaction}.

\begin{figure}[h]
    \centering
    \subfloat[Notation for interaction]{
        \includegraphics[width=0.45\textwidth]{Figures/NotationOfInteraction.pdf}
        \label{subfig:notation-interaction}
    }
    \subfloat[Relation between distance and force]{
        \includegraphics[width=0.45\textwidth]{Figures/physicalinteraction.pdf}
        \label{subfig:interaction-relation}
    }
    \caption[Interaction between two
    pedestrians.]{Figure \subref{subfig:notation-interaction} shows that the interaction between two pedestrians is governed by the distance between pedestrians $d_{\alpha \beta}$, the unit vector $\overrightarrow{u_{\beta \alpha}}$ pointing from $\beta$ to $\alpha$, and the angle $\theta$ between $\alpha$'s direction of movement and the vector pointing from $\alpha$ to
    $\beta$.\\
Figure \subref{subfig:interaction-relation} shows the repulsive force away from $\beta$ increases exponentially the closer $\alpha$ gets to $\beta$.} \label{fig:pedestrian-interaction}
\end{figure}

The relation between the repulsive force from $\beta$ to $\alpha$ is seen in
figure~\ref{subfig:interaction-relation}. The force increases exponentially
the closer $\alpha$ gets to $\beta$. This means that the force will hardly be
noticeable for pedestrians that are far away from each other, but is strong
enough when they are close to each other to prevent them from colliding,
without adding any physical interaction forces. The threshold for when the
force becomes large enough to have an impact on the movement of $\alpha$ (the
range of the force) depends on the parameter $B$. The strength of the force
depends on the parameter $A$.

\begin{table}
    \centering
    \begin{tabular}{l l}
        \toprule
        \multicolumn{2}{c}{\textsf{Parameters of the interaction force}}\\
        $\theta_{\alpha \beta}$ & Angle between $\alpha$'s direction of
        movement and the vector from $\alpha$ to $\beta$\\
        $d_{\alpha \beta}$& Distance between pedestrians $\alpha$ and $\beta$ \\
        $\overrightarrow{u_{\beta \alpha}}$& Unit vector pointing from $\beta$ to $\alpha$ \\
        $\lambda$& Parameter controlling the anisotropic property of the
        interaction\\
        $A$& Parameter controlling the interaction strength \\
        $B$& Parameter controlling the range of the repulsive interaction  \\
        \bottomrule
    \end{tabular}
    \caption{Summary of parameters of the interaction force}
    \label{tbl:interaction-forces}
\end{table}

\subsection{Repulsion from the walls}
\label{sec:wall-repulsion}
The \emph{repulsive force} from the walls is the force that prevents
pedestrians form walking into, or even through, walls in the simulation. As
with the interaction force between pedestrians, the repulsive force from the
walls functions without taking into account the physical forces between walls
and pedestrians.

The repulsion from the wall $\gamma$ is given by:

\begin{equation}\label{wallpotential}
    \overrightarrow{f_{\alpha \gamma}}(t) =
    - \nabla_{\overrightarrow{p_{\alpha}}} h
    \left( d_{\alpha \gamma}(t) \right)
\end{equation}

Here $h$ is a repulsive potential and $d_{\alpha \gamma}(t)$ is the distance
from the pedestrian to the nearest point on the wall at time $t$.
$\nabla_{\overrightarrow{p_\alpha}}$ is the gradient of the repulsive
potential with respect to the pedestrian's position. The notation for the wall
interaction is summarised in figure~\ref{fig:wall-notation}.

It is not clear from the article how finding the nearest point on the wall is
accomplished. In most cases this can be done by projecting the position of the
pedestrian onto the line segment that defines a wall. Some complications
arise when walls are at odd angles to pedestrians; this is discussed further
in section~\ref{sec:repulsion-points}. In the following, we assume that the
nearest point on the wall, $\overrightarrow{p_{\gamma \alpha}}$, is known.

\begin{figure}[ht]
    \centering
    \begin{tikzpicture}
        \node (p) [pedestrian, label=below:$\alpha$] {};
        \node (pg) [marked point,above=of p,
                    label=above right:$\overrightarrow{p_\gamma \alpha}$] {};

        \node (wall start) [wall endpoint,left=3cm of pg] {};
        \node (wall end) [wall endpoint,right=2cm of pg] {};

        \draw [wall] (wall start)
            to node[font=\huge,xshift=-1.5cm, above left] {$\gamma$} (wall end);

        \draw [distance marker] (p.center)
            to node[right] {$d_{\alpha \gamma}(t)$} (pg.center);
    \end{tikzpicture}
    %{\includegraphics[width=0.5\textwidth]{Figures/NotationOfWall.pdf}}
    \caption[Notation for the interaction between pedestrians and
    walls]{Notation for the interaction between pedestrians and walls.
    $\overrightarrow{p_{\gamma \alpha}}$ is the point on the wall $\gamma$ nearest to pedestrian
    $\alpha$ and  $d_{\alpha \gamma}(t)$ is the distance between the
    pedestrian and this point.}
    \label{fig:wall-notation}
\end{figure}

The definition of $h\left( d_{\alpha \gamma}(t) \right)$ is not given in
\cite{self-org}. However, as with the interaction force between pedestrians,
it is given in \cite{ABconstant} by:

\begin{equation}
    h \left( d_{\alpha \gamma}(t) \right) =
    U e^{\frac{- d_{\alpha \gamma}(t) }{ R_{\alpha} }}
\end{equation}

where $U$ is a constant given as a model parameter.

Since $p_{\gamma \alpha}$ and $p_\alpha$ are known, the gradient can be
calculated only for the magnitude of the force, adding the direction later.
Given this interpretation, the wall interaction force becomes:


\begin{equation}
    f_{\alpha \gamma}(t) =
    -\frac{\partial}{\partial d_{\alpha \gamma}(t)}U e^{\frac{- d_{\alpha
    \gamma}(t) }{ R_{\alpha} }} \overrightarrow{u_{\gamma \alpha}}
\end{equation}

where $\overrightarrow{u_{\gamma \alpha}}$ is the unit vector pointing from
$\gamma$ to $\alpha$.

Differentiating this gives:

\begin{equation}
    f_{\alpha \gamma}(t) =
    \frac{U}{R_\alpha}
    e^{\frac{- d_{\alpha \gamma}(t) }{ R_{\alpha} }}
    \overrightarrow{u_{\gamma \alpha}}
    \label{eqn:wall-repulsion}
\end{equation}

Since the distance between the pedestrian and the wall cannot be negative, the
exponential term of the repulsion will always be between zero and one, rising
towards one as the distance decreases. This means that $0 < |f_{\alpha
\gamma}(t)| < U/R_\alpha$, increasing exponentially towards $U/R_\alpha$ as
the distance between the pedestrian and the wall decreases. Similar to the
interaction force between pedestrians, this means the repulsion from the walls
is negligeble when the distance is large, but strong enough to prevent
pedestrians from walking into or through walls even when no physical forces
are present. The model parameter, $U$, determines exactly how strong this
repulsive force is.

The parameters for the wall repulsion is summarised in
table~\ref{tbl:wall-repulsion}.

\begin{table}[h]
    \centering
    \begin{tabular}{l l}
        \toprule
        \multicolumn{2}{c}{\textsf{Parameters for wall repulsion}}\\
        $\overrightarrow{p_{\gamma \alpha}}$ & Point on the wall $\gamma$ closest to
        the pedestrian.\\
        $d_{\alpha \gamma}(t)$ & Distance from the wall $\gamma$ to pedestrian
        $\alpha$. \\
        $U$ & Constant determining the magnitude of the repulsive force from
        walls. \\
        \bottomrule
    \end{tabular}
    \caption{Summary of parameters for wall repulsion.}
    \label{tbl:wall-repulsion}
\end{table}

\subsection{Summary}
As we have seen, the model consists of three main parts: the desired movement
force, the interaction force between agents and the repulsive force from
walls. The desired movement force represents the pedestrians desire to move at
a certain speed, given as a model parameter. The force accelerates the
pedestrian to this speed, even allowing the pedestrian to exceed this speed to
catch up to lost time in the event of a slowdown. The interaction force
prevents pedestrians from walking into each other. It is a repulsive force
between pedestrians that decreases exponentially with the distance to each
other. Model parameters determines the magnitude and range of the force.
Finally, the repulsive force from the walls prevents pedestrians from bumping
into and walking through walls. This force, like the interaction force,
decreases exponentially with distance, and its magnitude is determined by a
model parameter.

A complete summary of the model's notation and parameters is given in
table~\ref{tbl:model-summary}.

Our review of the model has revealed various areas that we need to address in
order to create practical numerical simulations. We have already filled out
some gaps by adding details from articles other than the original, but the
model as we have presented it in this article is still quite abstract. In the
following section, we outline the parts missing to turn the model into a
simulation, and how we have solved these problems.


\begin{table}[h]
    \centering
    \begin{tabular}{l l}
        \toprule
        \multicolumn{2}{c}{\textsf{Main forces}}\\
        $\overrightarrow{f_{\alpha}^{d}}$ & Desired movement force\\
        $\overrightarrow{f_{\alpha \beta}}$ & Interaction force from
        pedestrian $\beta$\\
        $\overrightarrow{f_{\alpha \gamma}}$ & Repulsive force from wall
        $\gamma$\\

        \midrule
        \multicolumn{2}{c}{\textsf{Base notation}}\\
        $\overrightarrow{p_{\alpha}}$ & Position vector of pedestrian
        $\alpha$\\
        $\overrightarrow{V_{\alpha}}$ & Velocity of pedestrian $\alpha$ \\
        \addlinespace[0.3em]
        $R_\alpha$ & Radius of pedestrian $\alpha$\\

        \midrule
        \multicolumn{2}{c}{\textsf{Parameters of the desired movement force}}\\
        $\overrightarrow{V_{\alpha}}(t)$ & Velocity of pedestrian $\alpha$
        at time $t$\\
        $V_{\alpha}^{d}(t)$ & Desired speed of pedestrian $\alpha$ at time
        $t$\\
        $V_{\alpha}^{Id}$ & Initial desired speed of pedestrian $\alpha$ \\
        $\langle V_{\alpha}(t) \rangle$ & Average speed of pedestrian
        $\alpha$ \\
        $\tau$& Relaxation time \\

        \midrule
        \multicolumn{2}{c}{\textsf{Parameters of the interaction force}}\\
        $\theta_{\alpha \beta}$ & Angle between $\alpha$'s direction of
        movement and the vector from $\alpha$ to $\beta$\\
        $d_{\alpha \beta}$& Distance between pedestrians $\alpha$ and $\beta$ \\
        $\overrightarrow{u_{\beta \alpha}}$& Unit vector pointing from $\beta$ to $\alpha$ \\
        $\lambda$& Parameter controlling the anisotropic property of the
        interaction\\
        $A$& Parameter controlling the interaction strength \\
        $B$& Parameter controlling the range of the repulsive interaction  \\

        \midrule
        \multicolumn{2}{c}{\textsf{Parameters for wall repulsion}}\\
        $\overrightarrow{p_{\gamma \alpha}}$ & Point on the wall $\gamma$ closest to
        the pedestrian.\\
        $d_{\alpha \gamma}(t)$ & Distance from the wall $\gamma$ to pedestrian
        $\alpha$. \\
        $U$ & Constant determining the magnitude of the repulsive force from
        walls. \\
        \bottomrule
    \end{tabular}
    \caption{Summary of the model.}
    \label{tbl:model-summary}
\end{table}

