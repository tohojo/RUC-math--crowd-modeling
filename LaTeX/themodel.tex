% vim:ft=tex
\section{The model}
\label{sec:the-model}
%	There should be an introduction here. What will the reader learn during the section. 	%


\subsection{Explanation of our model}
In this section we will go through the model from the article \cite{self-org} in great detail. 
First of all,  as stated briefly earlier this model is an agent based social force model. 
This means that the model uses individual entities to say something about the crowd as a whole. 

Each entity or agent is acted on by a collection of different forces, and the agent cannot escape 
from Newton's three laws about the motion of any object, which mainly says that when there is a 
force it results a corresponding acceleration.  However, our agent is not a ball being kicked around, 
it has a willingness to go to some destine place.  The way that agent implement the will is by 
generating a statistic frictional force from the ground, and that force is here what we called 
the social force.

These forces can be either repulsive or attractive. The general approach of the model 
is fairly simple. An agent $\alpha$ wants to go in a desired direction with a desired speed. 
However the environment and other agents might force agent $\alpha$ to stray from the desired 
direction or have speeds different from the desired one. The change in position of 
agent $\alpha$ is given by:

\begin{equation}
		\frac{d \vec{r_{\alpha}}}{dt} = \vec{V_{\alpha}} \left( t \right)
\end{equation}

And, as we know from newtonian physics, the acceleration of agent $\alpha$ is 
then given by the summation of all the forces acting on the agent:

\begin{equation}
    \frac{d \vec{V_{\alpha}}}{dt} = \vec{f_{\alpha}} \left( t \right) 
\end{equation}

$\vec{f_{\alpha}} \left( t \right)$ is a summation of all the forces action in 
agent $\alpha$ from the environment and other agents. More specifically it is given by:

\begin{equation}\label{model}
    \vec{f_{\alpha}} = \vec{f^{0}_{\alpha}}\left( \vec{V_{\alpha}} \right) + 
    \vec{f_{\alpha B}} \left( \vec{r_{\alpha}} \right) +
    \sum_{\beta \neq \alpha} \vec{f_{\alpha \beta}} \left(\vec{r_{\alpha}}, 
    \vec{V_{\alpha}}, \vec{r_{\beta}}, \vec{V_{\beta}} \right) +
    \sum_{i} \vec{f_{\alpha i}} \left( \vec{r_{\alpha}}, \vec{r_{i}}, t 
    \right)
\end{equation}

So this is a summation of four different kinds of forces. We will go through 
them one at a time explaining their role in the model and their mathematical 
structure.\\

\begin{figure}[hb]
    \centering
    {\includegraphics[scale=0.35]{Figures/NotationOAgent.pdf}} 
    \caption{Shows a visual presentation of the mathematical notation.}
    \label{NotationOAgent}
\end{figure}

\subsubsection{The velocity dependence} %gotta figure out a better name for this part
The first term on the right hand side of equation \eqref{model} is a velocity dependent force 
and it is given by:

\begin{equation}
	\vec{f^{0}_{\alpha}}\left( \vec{V_{\alpha}} \right) =
    \frac{1}{\tau}
    \left( V_{\alpha}^{0} \vec{e_{\alpha}} - \vec{V_{\alpha}} \right)
\end{equation}

Here $\tau$ is the relaxation time. % lets go a little more into detail about this relaxation time. We discussed it with viggo last time
$\vec{V_{\alpha}}$ is the  current velocity of the agent and $V_{\alpha}^{0}$ 
is the initial speed, that is the speed of the agent at the end of the last 
simulation step. $V_{\alpha}^{0}$ is given by:

\begin{equation}\label{v0}
    V_{\alpha}^{0} = \left[ 1 - \eta_{\alpha} \left( t \right) \right] 
    V_{\alpha}^{0} \left( 0 \right) +
    \eta_{\alpha} \left( t \right)V_{\alpha}^{\text{max}}
\end{equation}

Here $V_{\alpha}^{0} \left( 0 \right)$ is the velocity at the beginning of the 
first simulation step and $V_{\alpha}^{\text{max}}$ is the desired speed of agent
$\alpha$, that is the speed that agent $\alpha$ will try to get if it is allowed by the 
environment and other agents. $V_{\alpha}^{\text{max}}$ can be exceeded. $\eta_{\alpha}$ 
is called the impatience or nervousness of the agent and is given by:

\begin{equation}
	\eta_{\alpha} \left( t \right) =
    1 - \frac{\overline{V}_{\alpha} \left( t \right)}
             {V_{\alpha}^{0} \left( t \right)}
\end{equation}

Here $\overline{V}$ is the average speed in the desired direction and as 
earlier $V_{\alpha}^{0} \left( 0 \right)$ is the speed at the beginning of the 
first calculation step of the simulation. $\eta_{\alpha}$ is called the impatience 
or nervousness factor of agent $\alpha$ and we will a little time on it her because 
it does yield some interesting dynamics. 

First of all The impatience or nervousness factor is active when one calculates the 
force action on agent $\alpha$ from the velocity of the agent.

In the case where $0 \leq \eta_{\alpha} \leq 1$ the expression for 
$V_{\alpha}^{0} \left( t \right)$  makes sense. Here we can see why this term 
is called the impatience of the agent. If the fraction  between the average 
speed in the desired direction and the initial speed is low then $\eta_{\alpha} \approx 1$. 
When the impatience term is close to one $V_{\alpha}^{0} \left( t \right)$ 
is dominated by $V_{\alpha}^{\text{max}}$. That is, if the agent have not 
moved very far in the desired direction compared to the initial speed the 
impatience of the agent will cause the agent's future velocity to be dominated by 
the desired velocity of the agent.

If the agent has been moving in the desired direction with his initial 
speed the entire time then $\eta_{\alpha} = 0$  and 
$V_{\alpha}^{0} \left( t \right)$ will continue to be $V_{\alpha}^{0} \left( 0 \right)$.

In the case where $\eta_{\alpha} \leq 0$ that is the agent has moved further 
in the desired direction then he would have had he been walking with his 
initial speed. The expression for $V_{\alpha}^{0} \left( t \right)$
stats yield strange results. That $\eta_{\alpha} \leq 0$ would imply that:

\begin{equation}\label{n}
    V_{\alpha}^{0} = \left[ 1 + \eta_{\alpha} \left( t \right) \right] 
    V_{\alpha}^{0} \left( 0 \right) -
    \eta_{\alpha} \left( t \right)V_{\alpha}^{\text{max}}
\end{equation}

And this will yield a negative value for $V_{\alpha}^{0}$ if: 

\begin{equation}
\left[ 1 + \eta_{\alpha} \left( t \right) \right] 
V_{\alpha}^{0} \left( 0 \right) < \eta_{\alpha} \left( t \right)V_{\alpha}^{\text{max}} 
\end{equation}

This is a problem because it is not that far fetched that an agent will be 
forced to exceed his desired velocity.

In the case where $1 \leq \eta_{\alpha}$ it would mean that the agent has moved 
further in the opposite direction than the desired one and this can only happen very 
weird situations.

Equation (\ref{n}) and Equation (\ref{v0}) contains an intermediate variable $ n_{\alpha} \left( t \right) $, 
so in principle we are allowed to eliminate $ n_{\alpha} \left( t \right) $ and only show the 
relationship between $ V_{\alpha}^{0}(t) $ and $ \overline{V}_{\alpha} \left( t \right) $. Thus we get:

\begin{equation}\label{vv}
    V_{\alpha}^{0}(t) = \left[ 1 - \frac{v_{\alpha}^{max}}{v_{\alpha}^{0}(0)}\right]\overline{V}_{\alpha} \left( t \right) + v_{\alpha}^{max}
\end{equation}

\begin{figure}[ht]
\centering
{\includegraphics[scale=0.35]{Figures/impatience.pdf}} 
\caption{\small{The function about the desired velocity $ V_{\alpha}^{0}(t) $ 
and the average velocity in the desired direction of motion $ \overline{V}_{\alpha} \left( t \right) $}}
\label{impatience}
\end{figure}

Figure (\ref{impatience}) is a drawing of the graph about those two variables, and the intersection
 of the function line with both axis are:

\begin{equation}
\left( 
	\overline{V_{\alpha}} , V_{\alpha}^{0} \left( t \right)
\right)
=
\left( 
	0 
		, 
	v_{\alpha}^{max} 
\right) 
\text{and} 
\left(
	v_{\alpha}^{max} 
		\frac{v_{\alpha}^{0} \left( 0 \right) }{v_{\alpha}^{max}-v_{\alpha}^{0} \left(0 \right)} 
	, 0 
\right) 
\end{equation}
Now there is a doubt whether the value of $ \overline{V}_{\alpha} \left( t \right) $ can reach $ v_{\alpha}^{max}\frac{v_{\alpha}^{0}(0)}{v_{\alpha}^{max}-v_{\alpha}^{0}(0)} $, if we have already 
set $ v_{\alpha}^{max} $ a fixed number for a certain agent. In the case:

\begin{equation}
	v_{\alpha}^{max} 
	\geq 
	v_{\alpha}^{max} 
	\frac{v_{\alpha}^{0}(0)}{v_{\alpha}^{max}-v_{\alpha}^{0}(0)}
\end{equation}

we get the relation:

\begin{equation}
v_{\alpha}^{0}(0)\leq \frac{1}{�2}v_{\alpha}^{max}
\end{equation}

% Lets have a little summation here. What have we learned about the inpatience factor and
% the velocity dependant force. What kind of dynamics does this force yield

\subsubsection{Interaction with the walls}
Now the second term on the right hand side of \eqref{model} is the forces acting on agent 
$\alpha$ from the walls of the room. The force is given by:

\begin{equation}
    \vec{f_{\alpha B}} \left( \vec{r_{\alpha}} \right) =
    - \nabla_{\vec{r_{\alpha}}} V_{B}
    \left( \| \vec{r_{\alpha}} - \vec{r_{B}^{\alpha}} \| \right)
\end{equation}

Here $\nabla_{\vec{r_{\alpha}}}$ is the gradient and $V_B$ is a repulsive 
potential. $ \| \vec{r_{\alpha}} - \vec{r_{B}^{\alpha}} \|$ is the distance 
from agent $\alpha$ to the nearest point of the nearest wall.

\begin{figure}[ht]
\centering
{\includegraphics[scale=0.35]{Figures/NotationOfWall.pdf}} 
\caption{\small{Shows the notation for the interaction with walls.}}
\label{NotationOfWall}
\end{figure}

The repulsive potential describes the force added from the walls to pedestrian $\alpha$, since $\alpha$ does not
want to get too close to the wall. So the closer $\alpha$ get to the wall the more the force from the wall gets.
The repulsion potential is given by:

\begin{equation}
V_{B} \left( \| \vec{r_{\alpha}} - \vec{r_{B}^{\alpha}} \| \right) =
V^0_{\alpha B} e^{- \| \vec{r_{\alpha}} - \vec{r_{B}^{\alpha}} \| / r_{\alpha} }
\end{equation}

Here $V^0_{\alpha B}$ is a constant and $r_{\alpha}$ is the radius of a pedestrian $\alpha$.

The reason for this repusion force from the wall is that the pedestrians do not want to get hurt by running into the walls
or get crushed between the panicing crowd and the wall [Helbing and Molnár, 1995]. %real references, please.

\subsubsection{Interaction between agents}
The third term on the right hand side of \eqref{model} is a summation of all the 
force between agent $\alpha$ and agent $\beta$. It is a function of the position vector and the velocity of 
both agents, and it is given by:

\begin{equation}
    \sum_{\beta \left( \neq \alpha \right)}
        \vec{f_{\alpha \beta }}\left( t \right) =
        A_{\alpha}^{1} exp \left(
            \frac{ r_{\alpha \beta} - d_{\alpha \beta }}
                 {B_{\alpha}^1}
        \right)
    \vec{\eta_{\alpha \beta}} \cdot
    \left(
        \lambda_{\alpha} + \left(
            1 - \lambda_{\alpha}
        \right)
		\frac{1+\cos{\phi}}{2}
    \right) +
    A_{\alpha}^{2} exp\left(
        \frac{r_{\alpha \beta} - d_{\alpha \beta}}
             {B_{\alpha}^{2}}
    \right)
    \vec{\eta_{\alpha \beta}}
    \label{agentinteraction}
\end{equation}

\begin{figure}[ht]
    \centering
    {\includegraphics[scale=0.35]{Figures/NotationOfInteraction.pdf}} 
    \caption{Shows the notation for the interaction between agents.}
    \label{NotationOfInteraction}
\end{figure}

Here $A_{\alpha}^{1}$, $A_{\alpha}^{2}$, $B_{\alpha}^{1}$, $B_{\alpha}^{2}$ 
and $\lambda_{\alpha}$ are all constants that can differ for each agent. 
$r_{\alpha \beta}$ is the sum of the radii of $\alpha$ and $\beta$ that is 
$r_{\alpha \beta} = r_{\alpha} + r_{\beta}$. $d_{\alpha \beta}$ is the 
distance from the center of mass of agent $\alpha$ and the center of mass of 
agent $\beta$ and is therefore given by $d_{\alpha \beta} = 
\|\vec{r_{\alpha}}\left( t \right) - \vec{r_{\beta}}\left( t \right) \|$.
$\eta_{\alpha \beta}$ is the normal vector pointing from $\alpha$ to $\beta$ 
and it is given by:

\begin{equation}
    \eta_{\alpha \beta} =
        \frac{\vec{r_{\alpha}}(t) - \vec{r_{\beta}}(t)}
             {\|\vec{r_{\alpha}}(t) - \vec{r_{\beta}}(t) \|}
\end{equation}

the angle $\phi$ in \eqref{agentinteraction} is the angle between the normal 
vector pointing from agent $\beta$ to $\alpha$ and the direction in which 
agent $\alpha$ is moving. Cosine to the angle is 

\begin{equation}
\cos \left( \phi \right)
	\left( t \right) 
		= 
	- \vec{\eta_{\alpha \beta}}
		\left( t \right) 
	\cdot 
\vec{e_{\alpha}}\left( t \right)
\end{equation}

Equation \eqref{agentinteraction} is divided into two terms. The first term on 
the right hand side reflects the agents tendency to stay at a certain distance 
from other agents. This part of the force is called the private sphere because 
the agent prefers to have some free space around him if possible. The radius 
of the private sphere can differ from agent to agent. The constant 
$A_{\alpha}^{1}$, $B_{\alpha}^{1}$ and $\lambda_{\alpha}$ control the nature 
of the private sphere $A_{\alpha}^1$ and $B_{\alpha}^1$ control the strength 
and range of the interaction respectively. $\lambda_{\alpha}$ is there to take 
into account a persons tendency to focus on things happening in front of him 
rather than behind him.	% we should make a drawing of this.

The second term of equation \eqref{agentinteraction} deals with physical interaction.
In the situation where the density of the crowd is high the agents will have be closer
to each other and the social sphere is undermined. % go more into detail here 
So if we look away from the social sphere for a minute and concentrate on the physical
interaction we will see that if we omit the social sphere the calculation will be reduced to
to:

\begin{equation}\label{re}
\overrightarrow{f_{\alpha\beta}}(t) = A_{\alpha}^{2} exp\left[ \frac{r_{\alpha\beta} - d_{\alpha}\beta}{B_{\alpha}^{2}}\right]  \overrightarrow{n_{\alpha\beta}}
\end{equation}

Taking the norms of both sides of Equation (\ref{re}), we can draw the relation between the value of $\overrightarrow{f_{\alpha\beta}}(t)$ and $ d_{\alpha\beta} $, as shown in Figure 
(\ref{physicalinteraction}).\\

\begin{figure}
    \centering
    {\includegraphics[scale=0.45]{Figures/physicalinteraction.pdf}} 
    \caption{The function about the interaction force 
        $f_{\alpha\beta}(t)$ and the distance between two agents
        $d_{\alpha \beta}$ }
    \label{physicalinteraction}
\end{figure}

There is one intersection of the graph and the  axis at:

\begin{equation}
	\left( d_{\alpha \beta} , \| \vec{f_{\alpha \beta}} \left( t \right) \| \right)
 =
	\left( 0 , A_{\alpha}^{2} exp\left( \frac{r_{\alpha\beta} }{B_{\alpha}^{2}}\right)  \right) 
\end{equation}

If put into the constants, we will be able to get a maximum value of $ f_{\alpha\beta}(t) $, 
since the distance between agents cannot be negative. Here we set $ A_{\alpha}^{2} = 3 m/s^{2} $, 
$ r_{\alpha\beta} = 0.6 m $, and $ B_{\alpha}^{2} = 0.2 m $, so 
$ f_{\alpha\beta}(t)^{max} \doteq 60 m/s^{2} $, which is about six times the gravitational 
acceleration and represents a rather large force between agents (as large as six person's weight).

However, we notice that the effective part of the force calculated above is only the horizontal 
component that enables the agent to move horizontally in the plane where we do the simulation, 
but the reality is that the agents sometimes are also able to move vertically, for example, 
by stepping upon other people when they cannot take the pushing force from the surrounding agents. 
When that happens, the horizontal component of the repulsive force becomes smaller even if $ d_{\alpha\beta} $ 
is kept the same.	

\begin{figure}[ht]   
\centering
    {\includegraphics[scale=0.35]{Figures/ForceOverlapping.pdf}} 
    \caption{}
    \label{forceoverlapping}
\end{figure}

Therefore, a qualitative modification of dependence between $ f_{\alpha\beta}(t) $ and $ d_{\alpha\beta} $ could be:
% remember to finish this section

% again lets  have a little summation here. What kinds of dynamics does the
% social interaction part of the model yield.

\subsubsection{The attraction forces}
The fourth and last term in \eqref{model} represents the force from attraction 
in the room. Attractions can be either be either interesting sculptures or 
sights or familiar persons the agent prefer to be close to, such as friends 
and family. The mathematical structure of this force is the same as the force 
from other agents, however it is opposite in algebraic sign and has different 
constants. 

To get an overview of how the model is put together look a figure \ref{overview}
with the aid of table \ref{tableofconstandvar}

\begin{figure}[hb] %with some more comments i think that this figure could serve as a summation of the entire section
    \centering
    {\includegraphics[scale=0.45]{Figures/overview.pdf}} 
    \caption{An overview of how the model is put together}
    \label{overview}
\end{figure}

\clearpage
\section{Discussion}\label{sec:discussion}
\subsection{Limitations of the model}

\subsection{The force at doorways}

\subsection{Discussion on walls in special cases.}
In the general case of the repulsive force on a pedestrian, $\alpha$, from a wall nearby is given as a function of the vector from the nearest point. This point we calculate by finding the point that makes the vector form $\alpha$ to the wall be perpendicular to to vector that is the wall. In some cases though the point won't be on the wall it self. This of course makes no sense since you would then be repulsed by a non existing part of the wall meaning that you would avoid free areas which makes no sense. In this case you would have to use the end point of the wall. But doing this can make some unrealistic behavior as well, if the walls have the right composition. 

Let's start out by looking at a case with no problem. A case with no problems is a room where the angles between the walls is less than $180^o$, i.e. a squared room where they are $90^o$. For a pedestrian close to the corner between two walls, you would calculate the repulsive force from both of the walls. This you do in order for the pedestrian to avoid going through either one of the walls. When you do this you get a force directly away from each of the walls. This clearly makes sense and there is no problem in doing so.

\begin{figure}
\centering
\includegraphics[scale=0.5]{figures/Thewall.pdf}
\caption{Her you can se two walls joint together with an angle greater $180^0$ when seen from below. In the area `A` a pedestrian will only be perpendicular the first wall. In `B` there is no points perpendicular to any of the walls. And in `C` it is only the second wall that har points perpendicular to the pedestrians.  }
\label{fig:wallcase}
\end{figure}

This case were the angle between two walls is greater than $180^o$ could on the other hand give some problems if not handeled correctly. The case is sketched in figure \ref{fig:wallcase}. Here the are 3 different areas that a pedestrian $\alpha$ can be in. The area A where $\alpha$ is only perpendicular to wall $1$, in area B, $\alpha$ will not be perpendicular to any of the walls and in C he will be only perpendicular to wall 2. If a pedestrian is in area B then we would calculate the forces from the end point of the walls. This will be from the point where the two walls meet together. This will give you a double repulsion from one point and that doesn't make sense. Also when you are in are A or C you would get a repulsive force from a second wall you would be of no risk of going into and in many situations couldn't see because the first wall is blocking the sight. This of course doesn't make any sense too. So the way that we handle this situation is the following. When the angle between the walls is greater than $180^o$, from a pedestrian $\alpha$ point of view, you should look at the two walls as one, in the way that you will only calculate one force from the walls. In area A or C only the closest point on the closest wall should affect you. In the case of $\alpha$ being in area B the walls themselves   doesn't matter, only the vector going from the conjoint point of the walls to $\alpha$, should affect and only one time. Doing this, there should be no unrealistic scenarios.


\subsection{The repulsive force between agents in $ \Re ^{3}$}
From the given formula for calculating the repulsive force between agents in the description of the model, the part calculating the force to keep the personal space can be omitted when the agents are rather close to each other, then the calculation can be reduced as Equation (\ref{eq:re}).

\begin{equation}\label{eq:re}
\overrightarrow{f_{\alpha\beta}}(t) = A_{\alpha}^{2} exp\left[ \frac{r_{\alpha\beta} - d_{\alpha}\beta}{B_{\alpha}^{2}}\right]  \overrightarrow{n_{\alpha\beta}}
\end{equation}

Taking the norms of both sides of Equation (\ref{eq:re}), we can draw the relation between the value of 
$\overrightarrow{f_{\alpha\beta}}(t)$ and $d_{\alpha \beta}$, as in Figure (\ref{fig:physicalinteraction})
\\
\begin{figure}
\centering
\includegraphics[scale=0.45]{Figures/physicalinteraction.pdf} 
\caption{The function about the interaction force $\vec{f_{\alpha\beta}}(t)$ and the distance between two agents
$d_{\alpha\beta}$ }\label{fig:physicalinteraction}
\end{figure}

There is one intersection of the graph and the axis $ \left( 0, A_{\alpha}^{2} exp\left( \frac{r_{\alpha\beta} }{B_{\alpha}^{2}}\right)  \right)  $. If put into the constants, we will be able to get a maximum value of $ f_{\alpha\beta}(t) $, since the distance between agents cannot be negative. Here we set $ A_{\alpha}^{2} = 3 m/s^{2} $, $ r_{\alpha\beta} = 0.6 m $, and $ B_{\alpha}^{2} = 0.2 m $, so $ f_{\alpha\beta}(t)^{max} \doteq 60 m/s^{2} $, which is about six times the gravitational acceleration and represents a rather large force between agents (as large as six person's weight). \\\\
However, we notice that the effective part of the force calculated above is only the horizontal 
component that enables the agent to move horizontally in the plane where we do the simulation, 
but the reality is that the agents sometimes are also able to move vertically, for example, 
by stepping upon other people when they cannot take the pushing force from the surrounding agents. 
When that happens, the horizontal component of the repulsive force becomes smaller even if 
$d_{\alpha\beta}$ is kept the same.	
Therefore, a qualitative modification of dependence between $ f_{\alpha\beta}(t) $ and $ d_{\alpha\beta} $ could be:
\begin{figure}[hb]   
\centering
    {\includegraphics[scale=0.35]{Figures/ForceOverlapping.pdf}} 
    \caption{}
    \label{forceoverlapping}
\end{figure}
\\
\subsection{Use social force in further calculation}
use the value of forces to predict, as they are partly not real forces, the measurement does not reflect the reality in some range.
Pressure

